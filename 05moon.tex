\cleardoublepage
\chapter{The giant impact or a model for Moon formation}
\label{ch05}
\graphicspath{{./05figs/}}

\section{The formation of the Moon}
Despite its relatively close proximity compared to other astronomical objects, the formation of the Earths Moon presented a mystery deep within the space-age. In the 1960ies, sample-return missions like Apollo or Luna brought geological insight into the Moon. While not directly shedding light on the formation itself, these lunar samples showed that the Earth and the Moon have an identical signature in oxygen isotopes down to measuring uncertainty, opposing the heterogeneity in isotopic signatures in the solar system \citep{Wiechert:2001p3543}. This suggests, that the Moon has formed from the same material as the Earth. Another constraint on a formation theory of the Moon, is a relative depletion in iron. Seismic measurements on the Moon and also its bulk density suggest, that the Moon has an iron core with lass than $5 \%$ of its overall mass \citep{2011Sci...331..309W}. Another dynamical constraint on a theory for lunar formation is the conservation of the Earth-Moon systems angular momentum since its formation within $10\%$ of its initial value \citep{Canup:2001p3295}. 

Until the mid 1970ies, only three formation models have been known, all of them with their shortcomings \citep{Stevenson:1987p3540}. The fission theory goes back to the 19th century and proposes, that the Earth gained angular momentum due to some unknown mechanism and the rotation period went below the limit for rotational stability ($< 2.1h$). Part of the Earth was then shed into Earth orbit and re-accumulated into a satellite, which became our Moon. While this theory matches nicely the composition constraints, this formation mechanism would leave the Earth-Moon system with an angular momentum three times higher than todays value. Another theory suggests, that the Moon is simply a body which formed elsewhere and was later on captured into an orbit around the Earth. Aside from it's dynamical implausibility, it fails to explain the Moons iron depletion. A third theory proposes the co-accretion of the Moon in parallel to the Earth. While similar models might work well for the Giant planets moons, it fails to explain the relatively high angular momentum of the Earth-Moon system.

\section{The giant impact theory: Review of past work}
A more promising approach was the \emph{giant impact} theory first proposed by \cite{1975Icar...24..504H} and \cite{1976LPI.....7..120C}. They suggested, that todays Moon is a left-over of a giant impact between the forming Earth and an impacting second largest body from the proto-Earth's vicinity. The impact leaves debris in an Earth orbit, which later on accretes into the Moon. This model not only explains the high angular momentum of the Earth-Moon system, but also permits impact processes to account for the iron depletion of the Moon.

While \cite{1975Icar...24..504H} put the impactor size on a rather small $R \approx 1200km$ ($\approx 0.0048 \ME$ for a density of $4 g/cm^3$) based on a simple energy budget, \cite{1976LPI.....7..120C} advocated for a larger body in the Mars mass regime ($\approx 0.1 \ME$). Despite those estimates, the model remained qualitative and lacked quantitative verification. It was not until 9 years later, when the theory spur major interest at the \emph{Origin of the Moon} Kona conference in 1984. The advancement in numerical techniques, most notably SPH, and the now available computing power meant, that the theory could be put a for a test. 

In terms of time scales, the giant impact theory can be split up into three distinctive phases: The first phase is the actual impact. This happens on the order of the collisional timescale (see section \ref{ch03_sec01}) which yields $\approx 24h$ in case of the proto-Earth. Debris is put in orbit around the Earth. During a second phase, this debris accretes into individual moonlets, a process happening on $50 - 1000 \tau_K$ \footnote{$\tau_K$ denotes the Kepler time at the Roche limit for lunar density material of the proto-Earth ($\approx 2.9 \RE$) and yields $\approx 7h$} corresponding to a few months. In a third and final phase, these moonlets then undergo orbital evolution over millions of years due to tidal interaction with the proto-Earth and under circumstances even might collide if there were more than one \cite{Canup:1996p3541}.

The first direct simulations \citep{Benz:1985p1755, Benz:1989p1893, Cameron:2000p1854} of the actual impact not only showed that the basic mechanism of the giant impact actually work, but also delivered first quantitative estimates on the required impact parameters for successfully putting enough material into Earth orbit. Already \cite{1975Icar...24..504H} noted that about two lunar masses ($2 \ML = 0.024 \ME$) of silicates in orbit around the proto-Earth are required to later on successfully form the Moon. Similar results were obtained by N-body simulations of the accretion phase of the giant impact theory: \cite{Canup:1996p3541} found that either a bit more than a lunar mass outside the Roche limit or more than two lunar masses near or inside the Roche limit are sufficient for lunar accretion. Additionally they found that when most of the disk mass lies outside the Roche limit after the impact, multiple moonlet formation occurs. In this case tidal torques will change the orbit of the heavier moonlets more rapidly. So if the moonlet mass decreases with distance from the proto-Earth, the heavier moonlets will migrate outwards faster and outrun the lighter moonlets, ultimately colliding with them and forming one large Moon. \cite{Ida:1997p3395} performed a similar study and found scaling laws for the final moon mass in as a function of the mass and angular momentum of the proto-lunar disk. They showed, that the disk material can accrete into a Moon with an efficiency of $15 \% \dots 40 \%$, depending on the mass distribution and the angular momentum in the disk. Similar work by \cite{Kokubo:2000p2195} included a better collision handling in their N-body code and found similar results. The scaling laws from both studies agree well with the early results from \cite{Canup:1996p3541} and show that disk masses of $\approx 1.5 \dots 2.0 \ML$ are required to successfully accrete a Moon mass satellite.

The conundrum giant impact simulations have to solve is to find suitable impact parameters, leading to the desired post-impact configuration of the disk. Giant impact simulations require at certain resolution, in order to resolve the proto-lunar disk sufficiently. As collisions of bodies lack any inherent symmetry, such simulations have to be undertaken in 3D. For grid-based schemes resolution is defined spatially. This means that the hydrodynamics at the time of impact need to be sufficiently well resolved while still covering the entire proto-lunar disk after the impact. For a Lagrangian code like SPH, the resolution is given by the smallest mass element. With the proto-lunar disk containing only a few percent of the Earth-Moon systems total mass, a number of at least a few thousand particles is required in order to resolve the disk sufficiently.

The parameter space for the impact is comparable to the study performed in chapter \ref{ch03} of this thesis, with the four major parameters being target mass, mass ratio, impact angle and velocity: $\{\Mtar, \gamma, \thimp, \vimp\}$. Other parameters like composition of the colliding bodies were never considered to be free. Practically all simulations assumed, that both the target and the impactor are differentiated and of chondritic composition ($70 \wtp$ silicates and $30 \wtp$ iron core). In the 1980ies, parameter studies with an extent comparable to the one in chapter \ref{ch03}, compromised of several thousand simulations each using thousands of particles, were simply not feasible due to limitations in computing power. Early work \citep{Benz:1985p1755, Benz:1989p1893, Cameron:2000p1854} thus tried to constrain the parameter space as much as possible by certain assumptions and then only performed a handful of simulations located in the corresponding parameter subspace. 

A very strong constraint on the parameters is given by assuming mass conservation during the impact thus $\Mimp + \Mtar \approx \Mlr$, where $\Mlr$ is the mass of the largest remnant or in the particular case of the giant impact the Earth-Moon system. Depending on the timing of the giant impact during the proto-Earths accretion, the total mass is either chosen to be a fraction of or the whole mass of the Earth-Moon system $\ME + \ML \approx \ME$. For a given largest remnant mass, the mass conservation assumption thus gets rid of one free parameter. With mass conservation, also the angular momentum is conserved, so that $L_{imp} \approx 1.1 - 1.2 \lem$  \citep{Canup:2001p3295}. Note that this assumption is weaker, as even a small mass loss can lead to a considerable loss of angular momentum. The mass ratio is then constrained by the maximum angular momentum an impact can deliver at $\thimp = 90 \deg$ (or $b = 1$) where $\Limp = \Lgraz$. This puts a lower limit of $\gamma > 0.08 \dots 0.09$ \citep{Canup:2001p1861}. Under the assumption of mass conservation, the impact angle is for a given mass also determined by the angular momentum, as $\Limp = \Lgraz (\Mtar, \Mimp, \vimp) \sin{\thimp} \approx 1.1 - 1.2 \lem$. Thus the number of free parameters reduces to two. The impact velocity is also somewhat constrained, as high impact velocities lead to mass loss and would violate the initial assumption. 

\cite{Benz:1989p1893} used for the impact velocity values of $\vimp = \vesc$ ($\vneginf = 0$), but interestingly also tried out scenarios with $\vimp \approx 1.4 \vesc$ ($\vneginf = 10km/s$). They found that slow collisions ($\vimp = \vesc$) with an impactor mass of $\Mimp = 0.11 \dots 0.14 \ME$ and a total mass of $\Mtar + \Mimp = 1 \ME$ to be able to produce massive enough and iron-depleted proto-lunar disks. While the high velocity collisions also produced considerable disk masses, they were rejected because of a too large iron content (up to $1.3 \ML$). It is important to note that this simulations used only $3008$ particles and therefore resolved the disk with only $60 \dots 100$ particles! Determining the iron content of such a disk goes beyond the resolution. The iron percentage upper limit corresponds to only 5 iron particles and anything above that will be considered an iron excess. After this finding for high velocity collisions, later work never considered collisions with $\vimp > 1.10 \vesc$ again.

Simulations by \cite{1997Icar..126..126C} with considerably higher resolution had trouble reproducing the initial results by the series of papers by Benz. He found the optimum range for the mass ratio to be $\gamma = 0.25 \dots 1.0$ and the total mass of $\Mtot = 1 \ME$ in order to put enough material into orbit, but only with a rather high $\Limp \approx 2 \lem$. Thus he put forward an \emph{early scenario} in which the impact occurs when the Earth has accreted only part of its mass, so that $\Mtar \approx 0.5 \ME$. In \cite{Cameron:2000p1854} he showed with $10k \dots100k$-particles simulations, that a collision with $\{0.45 \ME, 7:3, 1.00 \vesc \}$ and a suitable impact angle to yield $L_{imp} \approx 1.1 \lem$ to be successful in  putting roughly two lunar masses into orbit with the right amount of final angular momentum. He also found the escaping mass to increase for slight increases of the impact velocity and used this as an argument for not considering further simulations with higher impact velocities. The main problem with this \emph{early impact} scenario remained that the Earth-Moon system still has to accrete $\approx 0.35 \ME$ until its final configuration. It is unlikely that the Moon would have kept its iron depletion during this post-impact accretion period.

\cite{Canup:2001p3295} still followed the path of the \emph{late impact} scenario and confirmed with $10k \dots100k$-particles simulations the same issue found by \cite{Cameron:2000p1854}: Late impacts only produce massive enough disks, when an excess angular momentum of $\Limp \approx 2 \lem$ is used. But their conclusion was different. As one possible explanation, \cite{Canup:2001p3295} state: \emph{If we assume that the Moon did in fact form via a large impact event, this finding suggest one or more of the following: (...) (2) regions of parameter space not explored in the above surveys could suggest different scaling relationships that would more easily yield the Earth-Moon system (...)}. Although \cite{Canup:2001p1861} were also unable to produce disks masses above $2 \ML$, they found that the satellite mass scaling law from  \citep{Kokubo:2000p2195} yields satellite masses above a Moons mass for impacts with $\{0.88 \ME, 0.1, \approx 53 \deg, 1.00 \vesc \}$. So far, this has been considered to be the most promising impact scenario for the giant impact. We call it the \emph{canonical model}.

Later work \citep{Canup:2004p115} concentrated on finding scaling laws for the disk mass in the low velocity regime. Similar to the early work by \cite{Benz:1989p1893}, they found that grazing collisions tend to produce iron-rich disks for $\vimp > 1.1 \vesc$ and thus also rejected the possibility of a high velocity impact. More recent work \citep{Canup:2008p3551} extended the parameter space by considering pre-impact rotation of both the target and the impactor. Interestingly, high velocity impacts up to $\vimp = 1.4 \vesc$ were investigated, although still with the same basic assumption of no mass escape. For that reason, these high velocity collisions were still performed with $\Limp \approx 1.1 \lem$, although the final angular momentum is lower with mass loss. Disk masses were found to be well below the required values for moon formation. The overall conclusion of this work was, that target pre-impact rotation might be favorable for Moon formation.

Impact simulations with other methods than SPH are sparse: The first simulations with a grid code incorporating hydrodynamics and self-gravity, were performed by \cite{Wada:2006p1013}. They performed a few simulations in the low-velocity regime comparable to previous SPH work. These grid code results agreed in general with the SPH results, despite only using very simple polytropic equations of state. This is very not surprising, as the collisions occur in the gravity regime and are largely dominated by gravitational processes. More recent work \citep{Canup:2010p3713} compared simulations performed with the Eulerian CTH code with the SPH results and also found the basic results to agree during the first few hours of the impact.

The main problem of the canonical model is the disagreement with the geo-chemical evidence: The canonical model predicts, that $70 \dots 80 \%$ from the disk material originates in the impactor. The isotopic similarity in oxygen isotopes between the Earth and the Moon thus heavily contradicts the observed heterogeneity in the solar system \citep{Wiechert:2001p3543}. A model has been put forward \citep{Pahlevan:2007p2065, 2011E&PSL.301..433P} which explains the isotopic similarity by re-equilibration of the oxygen isotopes through thorough mixing in the presumably hot proto-lunar disk after the impact event. It is still a qualitative model and lacks validation through direct simulations of the proto-lunar disk. A major problem of the model is the requirement, that all the silicate reservoirs mix with each other efficiently. Thermal inversion layers in the proto-Earth or clumps quickly being formed right after a accretion, a phenomenon often observed in simulations \citep{1997Icar..126..126C, Cameron:2000p1854, Canup:2001p1861, Canup:2004p115} would prevent efficient mixing and inhibit the individual reservoirs of oxygen to re-equilibrate.

\section{Alternative impact scenario}
The last section showed, that only a limited region of the parameter space for the giant impact has been studied so far. The fact that a successful model was found in the low velocity regime, does not exclude the possibility of other regions containing potential Moon-forming impacts. The search for new scenarios has been triggered for example by exotic models like \cite{2010M&PSA..73.5140W}, who suggest an \emph{icy impactor model} and argue that the impactors energy from beyond the ice line would provide much more kinetic energy available for the impact. The gain in kinetic energy for an impactor migrating from the solar system snow line ($\approx 2.7 AU$) to the Earths orbit leads to $\vneginf = 33 km/s$ or in case of a $\{0.9 \ME, 0.2\}$ impact to an impact velocity of $3.78 \vesc$. Such collisions are mainly disruptive and unlikely to lead to the formation of a relatively massive satellite. Nevertheless, it is important to check such alternative scenarios for viability with direct simulations.

\afterpage{
%\clearpage% To flush out all floats, might not be what you want
\begin{landscape}

\begin{figure}
\begin{center}
\includegraphics[scale=1.0]{02_disk_mass_c1.pdf}
\caption{Largest remnant total disk mass as a function of impact angle for different impact velocities for the \emph{c1} simulation set from the parameter study in chapter \ref{ch03}. Simulations with a largest remnant disk mass below $0.01\% M_{tot}$ are omitted.}
\label{ch05_fig02}
\end{center}
\end{figure}

\end{landscape}
}

Figure \ref{ch05_fig02} shows the largest remnants disk mass for the simulation set \css from the parameter study in chapter \ref{ch03}. Unlike the previous studies which only covered limited regios of the parameter space, this plot gives a complete overview of the parameter subspace for a given target mass and mass ratio. Although neither the target mass nor the mass ratio exactly match the values used in previous work and especially the canonical model by \cite{Canup:2001p1861}, the different velocity and impact angle regimes leading to considerable disk formation are recognizable. In case of $\{ 1 \ME, 0.1 \dots 0.2 \}$, a two lunar masses disk corresponds to $M_{disk} / \Mtot \approx 0.03$. The low velocity regime with $\vimp = 1.00 \dots 1.10$ can be clearly identified for grazing impact angles. Interestingly, for $\gamma \ge 0.2$ there exists a very narrow range of impact angles, for which also high velocity impact produce disks with a considerable mass. For example the largest remnant in $\{ 1 \ME, 0.1 \dots 0.2, 1.30 \vesc, 37.5 \deg \}$ has a disk with a mass of $M_{disk} \approx 0.05 \Mtot$ after the collision. It was noted in the previous section, that \cite{Benz:1989p1893} already performed collisions in a similar velocity regime and also noticed disk formation. But they rejected the case due to relatively high amounts of iron in the disks.

The simulations from the parameter study in chapter \ref{ch03} use a relatively small number of particles ($\approx 100k$). From looking at figure \ref{ch05_fig02}, the disk masses for the high velocity regime ($\vimp = 1.20 \dots 2.00$) is less independent on the actual parameters than in the low velocity regime ($\vimp = 1.00 \dots 1.10$). For $\gamma = 0.2$, the low velocity cases show considerable disk masses for a wide range of impact angles and also don't vary much between the individual target mass regimes. The high velocity regime is different: The $\{ 1 \ME, 0.1 \dots 0.2, 1.30 \vesc, 37.5 \deg \}$ case with a relatively massive disk for example appears to be an isolated case as a function of impact angle and appears only for $\Mtar = 1.0 \ME$. There narrow ranges for efficient disk formation occurs for the entire high velocity regime.

Qualitatively this can be explained by the basic mechanism behind disk formation: In order for material to end up in orbit around the largest remnant central clump, it requires just have the right orbital energy. If it's too low, the material is accreted onto the clump, if it's too high it escapes the remnant. In the low velocity regime, the disk is mainly compromised of the impactor. Its initial orbital energy is already near the limit for which material is gravitationally bound as $\vimp \approx \vesc$. Disk formation in this regime happens for relatively grazing angles, so most of the impactor material is only mildly decelerated and enters a bound orbit around the target or the later largest remnant. The required reduction in orbital energy is small compared to the range of orbital energy the material has to possess, so that it ends up in the disk. The contrary applies in the high velocity regime: The required reduction is large compared to the required range. It's harder to hit the right amount of orbital energy, as the allowed range is easily missed. This change is also more dependent on the actual impact geometry, as these collision occur at smaller angles than in the low velocity regime.

\begin{figure}
\begin{center}
\includegraphics[scale=0.5]{03_param_space_m0_010mimp.pdf}
\includegraphics[scale=0.5]{03_param_space_m0_015mimp.pdf}
\includegraphics[scale=0.5]{03_param_space_m0_020mimp.pdf}
\caption{Parameter space view of the simulations with chondritic impactor composition, each panel shows a different impactor mass. Every simulation is shown as a filled circle with an area proportional to the silicate disk mass $M_{disk, \silc}$. The black circle gives $2 \ML$ for reference. The color depicts the depletion factor in target material $\delta f_T$. Besides each circle, the bound angular momentum in $\lem$-units, the silicate disk and the iron disk mass in Moon masses are further noted.}
\label{ch05_fig03a}
\end{center}
\end{figure}

\begin{figure}
\begin{center}
\includegraphics[scale=0.5]{03_param_space_f0_010mimp.pdf}
\includegraphics[scale=0.5]{03_param_space_f0_020mimp.pdf}
\caption{Same parameter space plots as in figure \ref{ch05_fig03a} but for the simulations with an iron impactor ($70 \wtp$ Fe and $30 \wtp$ silicates.}
\label{ch05_fig03b}
\end{center}
\end{figure}

In order to investigate the high velocity regime more thoroughly, we performed a set of high-resolution simulations with half a million particles. The details and the results are described in detail in the manuscript below, but the covered parameter space will be quickly discussed to put into the context with the review of previous work. Figures \ref{ch05_fig03a} and \ref{ch05_fig03b} show a few of those simulations and their location in the parameter space.


mechanism of giant impact: ejection and second burn to circularize orbit

Different models for 

problem of parameter space, reference to chapter 3\\
constraints\\
assumptions made so far\\

additional work concerning
current situation\\

difficulties\\

sidenote: disagreement between different codes\\


refer to disk masses in chapter 3\\

%constraints:
%\cite{Wiechert:2001p3543}
%isotopic similarity within measuring uncertainty
%no more than 3\% H-chondrites or 5 \% Mars

\cite{Chambers:2001p2105}
median mass to hit the Earth $0.22 \ME$, median $\gamma = 0.67$
%\citep{Pahlevan:2007p2065}
%large isotopic heterogeneity in the solar system, resolve the issue of isotopic similarity between the earth and the moon (Wiechert), Moons depletion in volatiles


%\cite{2011E&PSL.301..433P} Pahlevan 2011

hot initial conditions:
\citep{2000orem.book..179P}

resolution dependence (clumping):
\citep{Canup:2010p3713} (SPH vs. CTH, 100k vs. 1M particles run still shows differences)

accretion process cannot be decoupled from the impact, as orbital evolution is already strong during the impact 

impact depends quite strongly on the equations of state, must be a geometry effect

sph might be problematic for integration times higher than a day due to numerical viscosity \citep{Canup:2004p115} 


%\section{\emph{Making the Moon from the Earths mantle} }
% new simulations

% additional figures
%TODOPLOT: disk thermal profiles (temp as color, ecc vs. rad)

%TODOPLOT: redo disk mass plot from c1 (linear scale)
%TODOPLOT: 


\cleardoublepage
\thispagestyle{empty}
\begin{flushright} \emph{manuscript to be published} \end{flushright}
{\Large Making the Moon from the Earths mantle} \vspace{0.5cm}\\
Andreas Reufer, Matthias M. M. Meier, Willy Benz and Rainer Wieler\\

The formation of the moon from a disk formed by a collision between the proto-Earth and a roughly Mars-sized impactor is widely accepted today. This \emph{giant impact} hypothesis \citep{1976LPI.....7..120C} explains the high angular momentum of the Earth-Moon system and the moon's deficiency in iron. Hydrodynamical simulations have shown that a differentiated impactor with a chondritic Si/Fe ratio hitting the proto-Earth at mutual escape velocity and a grazing impact angle of $45\deg$ can lead to a proto-lunar disk sufficiently massive to later form the Moon \citep{Canup:2001p1861}. Simulations of this \emph{canonical case} consistently show that only about $20\%$ of the mass of the (later) Moon is derived from the proto-Earth. Thus, it is surprising that lunar rocks show a strong elemental and isotopic similarity to the Earth's mantle, considering the elemental and isotopic heterogeneity of different solar system materials. Equilibration of a partially molten and partially vaporized post-impact disk with a terrestrial rock vapour atmosphere has been put forward \citep{Pahlevan:2007p2065} to explain this similarity. However, so far no complete, quantitative model of this equilibration exists \citep{2011E&PSL.301..433P}. Here we present simulations of a new class of collisions with higher impact velocities and a steeper impact angle. They result in an iron-poor Moon with the angular momentum of the Earth-Moon system and a considerably higher fraction ($50-80\%$) of proto-Earth material in the moon-forming disk. Shock heating during impact is much stronger than in the canonical case and the proto-lunar disk material is initially vaporized to a larger extent, possibly facilitating additional re-equilibration of the post-impact disk with the terrestrial mantle. We also investigate the influence of impactor composition (in silicon, iron and ice) on the outcome of the collision.

The collisional parameters for the giant impact considered so far have been limited to low-velocity collisions at, or only slightly above mutual escape velocity between target and impactor. The Earth-Moon system did not loose more than $10\%$ of its initial angular momentum ($1 \lem$) between the giant impact and today \citep{Canup:2001p3295}. In low-velocity collisions, very little mass and therefore also very little angular momentum is lost, compared to the total mass and angular momentum before the collision. Hence if impact velocity and angular momentum of the collision are fixed values, for a given impactor and target composition only one degree of freedom remains in the form of the product of the sine of impact angle and the impactor mass. Previous work therefore focussed on finding the optimum mass ratio between impactor and target. Most recent work suggests a mass ratio of 9:1 with a total mass of $1.05 \ME$  \citep{Canup:2004p115}. Both the impactor and the target are assumed to be differentiated bodies with a $30 \wtp$ iron core and a $70 \wtp$ silicate mantle. In all these low-velocity collisions the impactor loses kinetic energy whilst grazing past the target and is then dispersed into a disk around the target. As a result, the proto-lunar disk is composed mainly of impactor material. The fraction of target silicate to total silicate material in the disk
\begin{equation}
f_T = (\Mtar / M_{tot})_{disk}
\end{equation}
is only around $20\%$ where $\Mtar$ and $\Mtot$ denote the silicate part of the part of the disk originating in the target and total disk mass. If we define a similar target silicate material fraction for the post-impact Earth, we can deduce a depletion factor
\begin{equation}
\delta f_{T} = \frac{ (\Mtar / \Mtot)_{\textrm{disk}} }{ (\Mtar / \Mtot)_{\textrm{post-impact Earth} } }– 1
\end{equation}
which directly reflects the compositional similarity between the proto-lunar disk and the silicate portion of the Earth. This is important because the disk does not have to be derived completely from terrestrial (or target) material to match observations. It suffices that  the impactor contributes approximately the same fraction of material to the disk and the post-impact Earth. Therefore, it is this value $\delta f_{T}$ that has to approach $\approx 0\%$ (within the respective errors) to match geochemical observations.

These geochemical observations indicate that the composition of lunar rocks is best understood if their source material is either derived mainly from the Earth's mantle, or if the lunar accretion disk was thoroughly re-equilibrated with Earth's mantle in the aftermath of the collision. The oxygen isotope compositions of the Earth and the Moon are so close to each other ($\Delta \O17 \le 5 ppm$  \citep{Wiechert:2001p3543}) that, e.g, for an impactor that is as different in $\O17$ as Mars (Δ17O ~320 ppm), $\delta f_{T}$ can be no larger than about $\approx 5 \%$ (see Table \ref{ch05_tbl01} for the different values of $\delta f_{T}$ required by the different isotopic systems and impactor compositions). However, as there are at least two chondrite groups (enstatite and CI chondrites) that plot on the terrestrial mass-dependent fraction line in the oxygen three isotope diagram \citep{Clayton:1993p3544}, it is possible that the terrestrial composition is representative of a larger solar system reservoir, from which also the impactor could have been derived. A further constraint is the Si isotopic composition of Earth and Moon, which are also identical within error, but distinct from all chondrite groups, Vesta and Mars  \citep{2009E&PSL.287...77F, 2007Natur.447.1102G}. This was proposed to be the result of Si partitioning into the Earth's iron core, a process that is inefficient on planets that have less than $\approx 15-20\%$ of  Earth's mass \citep{2005E&PSL.236...78W}. Therefore, this distinct isotopic signature cannot have formed independently on the moon, or the impactor (unless the impactor was on the very heavy end of the mass range \citep{Canup:2004p115}, again linking the Moon to the terrestrial mantle. The lunar and terrestrial Mg/Cr ratios are similar ($\approx 100$ and $\approx 87$) but differ clearly from the chondritic value ($\approx 36$), which has also been explained by incorporation of Cr into Earth's core, a process that is inefficient on planets of the size of Mars \citep{2000orem.book..197J}. In addition, the $^{53} \textrm{Cr}/ ^{52}\textrm{Cr}$ ratio is identical within error for Earth and Moon, although it varies with heliocentric distance \citep{2000SSRv...92..225S}. Finally, Earth and Moon have a similar Hf/W-ratio and essentially identical $\varepsilon ^{182}\textrm{W}$ values \citep{2007Natur.450.1206T}. While in giant impact simulations both bodies may end up with an identical $\varepsilon ^{182}\textrm{W}$ value, in no case simulated by \cite{2010E&PSL.292..363N} were the observed Hf/W-ratios simultaneously matched. From all these observations, it was concluded \citep{2010E&PSL.292..363N, 2011E&PSL.301..433P} that either the moon had to be derived predominantly from the Earth's mantle, or that the post-impact disk and the Earth's mantle must have re-equilibrated. So far, the first of these two possible solutions was incompatible with the results of hydrodynamical simulations. In the following paragraphs, we present a new series of simulations, where for the first time a significantly higher fraction of the lunar material is derived from the Earth's mantle.

The collisions presented here fall into the broad regime of slow hit-and-run collisions \citep{Asphaug:2006p3729}, which has never before been considered for the giant impact. Because of the higher impact velocities ($1.2-4 \vesc$) in this type of collisions, substantial mass and angular momentum can be lost in the process. Therefore the initial angular momentum is less constrained and can be considerably higher than $1 \lem$. In hit-and-run collisions most of the impactor escapes, so that the disk fraction is elevated in target-derived materials. Table 1 in the supplementary online material shows a selection of around 60 simulations of this type performed by us. The higher impact velocities of these collisions are also encouraged by current models of terrestrial planet formation \citep{2006Icar..184...39O}. 

We used a standard SPH code with self-gravity and the ANEOS equation of state \citep{1972Thompson} for iron and water ice and M-ANEOS  \citep{Melosh:2007p3502} for silicate. All simulations used a resolution of around 500'000 particles and were performed in a similar manner as in previous work \citep{Benz:1985p1755, Canup:2001p1861, Canup:2001p3295, Canup:2004p115}. 

A reference case (\emph{cA08}) uses initial parameters and conditions comparable to those used in the canonical case \citep{Canup:2004p115} and successfully reproduces an iron-depleted proto-lunar disk massive enough to form a Moon. For each of our simulations, the Moon mass is calculated using the disk mass and the specific angular momentum \citep{Kokubo:2000p2195}. We found collisions with an angle of $30-40\deg$ and $1.2 – 1.3 \vesc$ are also successful to put target material into orbit, when using differentiated impactors with chondritic Si/Fe composition ($30 \wtp \textrm{Fe}, 70 \wtp \silc$) and masses between $0.15-0.20 \ME$. Some cases in this regime show an iron excess of $> 5 \wtp$ in the proto-lunar disk and are rejected, as in previous work \citep{Canup:2004p115}. The best chondritic case matching all constraints is obtained using an impact angle of $35 \deg$ and a velocity of $1.20 \vesc$, where $56\%$ of the material in the silicate part of the disk originates from the target and $\delta f_{T}$ reduces to $-35\%$ compared to $-66\%$ in the canonical reference case. Figure \ref{ch05_fig01a} shows four consecutive snapshots of such a high velocity collision.

The origin of the matter ending up in the proto-lunar disk is mainly defined by the geometry of the collision and is determined during the very early phase when the impactor is accelerating the target material. This can be seen in Figure \ref{ch05_fig01b} where the particles which later end up in the disk are highlighted in bright colour. In the canonical case, the impactor grazes around the target's mantle and is deformed. Due to the low impact velocity, material supposed to end up in orbit around the Earth must not be decelerated too strongly in order to retain enough energy to stay in orbit. In the canonical case, this is only the case for the parts of the impactor mantle most distant to the point of impact and some minor part of the target's mantle. But if impact velocity is increased from $1.00 \vesc$ (\emph{cA08}) to $1.30 \vesc$ (\emph{cC01}), parts from deeper within the target mantle receive the right amount of energy for orbit insertion. The outer regions of the target mantle receive too much energy and leave the system. This material removes mass and angular momentum from the system. While in case \emph{cB04} only $\approx 10\%$ of the initial angular momentum is removed, $\approx 45\%$ are removed in case cC06 (see table 1 in SOM).

If the collision geometry predominantly determines the fraction of target material in the proto-lunar disk, altering the size of the impactor by density changes should also change the target material fraction in the disk. A denser impactor with the same mass delivers the same momentum, while reducing “spill-over” of impactor material as it can be seen in figure \ref{ch05_fig01b}. High density, iron-rich impactors might be a product of composition changing hit-and-run collisions \citep{Asphaug:2010p3539}. We investigated this scenario with $50 \wtp$ and $70 \wtp$ iron core fraction impactors. With a $0.2 \ME$ impactor at $30 \deg$ impact angle and $1.30 \vesc$ impact velocity the target material fraction $f_T$ increases from $57\%$ ($\delta f_{T}$ = -34\%) in the chondritic case (\emph{cC01}), to $64\%$ ($\delta f_{T}$ = -28\%) in the $50 \wtp$ iron core case (\emph{fA01}) up to $75\%$ ($\delta f_{T}$ = -19\%) in the $70 \wtp$ iron core case (\emph{fB06}). However with the target material fraction, also the bound angular momentum and iron content of the disk increase to unrealistic values. Apparently, reducing the “spill-over” also reduces the lost mass and therefore lost angular momentum. 

We also looked into less dense but still fully differentiated impactors with a composition of $50 \wtp$ ice, $35 \wtp$ silicate and $15 \wtp$ iron, typical for small bodies accreted in regions of the solar system beyond the snow-line. Such a volatile-rich impactor scenario with an oxidized mantle is motivated  by terrestrial Pd-Ag and Cr, Sr isotope geochemistry data, suggesting the very late accretion of $10-20\%$ of CI-chondrite like material \citep{2010Sci...328..884S}. In this scenario, the efficiency of putting material into orbit is reduced compared to cases with denser impactors, though the fact that there is less impactor silicate material available to end up in the disk actually raises the target material fraction to $81\%$, and $\delta f_{T}$ to -10\% in the case of an $0.2 \ME$ impactor hitting at $30 \deg$ impact angle and $1.30 \vesc$ impact velocity. However the resulting silicate portion of the disk is not massive enough ($0.73 \ML$) to later form the Moon.  

The target material depletions yielded by the different models are compatible with geochemical constraints, with the possible exception of oxygen (see table \ref{ch05_tbl01}). They can also explain the difference in FeO content observed in the mantles of Earth and Moon (8wt\% and 13wt\% FeO, respectively[12]) by simple mixture of an impactor mantle with a certain FeO content and a proto-Earth with a FeO content of ~8wt\% (assuming this is the equilibrium value determined by the oxygen fugacity and core size \citep{2006mess.book..803R}). A typical $\delta f_{T}$ of -35\% for a 0.2 ME impactor then results in an impactor mantle FeO content of ~18\%, identical with Mars (see Figure 1 in SOM).

While the high velocity collision scenarios suit the dynamical and geochemical post-impact constraints as required for Moon formation, they differ considerably from the canonical case in terms of thermal history. The higher impact velocity leads to stronger shocks and therefore to more intense impact heating. Figure \ref{ch05_fig01c} shows cuts through the post-impact Earth after $\approx 58h$ with colour-coded temperature. In the canonical case (Figure \ref{ch05_fig01c} , left plot), only a thin blanket of material raining back from the disk experienced strong heating, while most of the mantle is only moderately heated above the initial average temperature of $2000K$ used in the simulation. In the high velocity case the shock is stronger and heats a substantial part of the mantle to temperatures above $10'000K$. Beside possible implications for the early history of the Earth, this also constitutes a different starting point for the re-equilibration model \citep{Pahlevan:2007p2065}, where the mantle of the post-impact Earth has to be fully convective in order to allow full isotopic re-equilibration with the post-impact disk. While in the canonical model the thin hot blanket may inhibit convection at early stages, the much thicker hot layer which extends deep into the mantle in our model probably simplifies exchange of material between the post-impact Earth and the hot silicate vapor atmosphere. Our results not only relax the required efficiency of re-equilibration by starting with a disk with a higher target material fraction, but the resulting thermodynamical state might actually create a more favorable starting point for re-equilibration.

Another key difference to the canonical model is the dynamical state of the proto-lunar disk. In both models, material is ejected on ballistic trajectories after impact and forms an arm structure. Angular momentum is transferred from the inner to the outer parts, thus circularizing the orbit of the outer parts of the arm while the inner arm re-impacts the Earth. In the high velocity cases, this arm structure persists for a smaller time compared to the canonical case, transferring less angular momentum and leaving the material on more eccentric orbits. The dynamical evolution of such an eccentric disk and the consequences for Moon formation have yet to be studied.

In summary, our new approach of a high velocity ($1.30 \vesc$) hit-and-run collision can explain the isotopic similarity of the Earth and the Moon for all isotope systems except possibly oxygen (see table \ref{ch05_tbl01}). This is a clear improvement over the canonical models \citep{Canup:2004p115}, where none of the known isotopic similarities could be explained. Oxygen only cannot be explained by our model if the impactor indeed was as different in oxygen isotopes as Mars. In that case, a re-equilibration model \citep{Pahlevan:2007p2065, 2011E&PSL.301..433P} might explain the similarity. Our work then also constitutes a new, hotter starting point for this re-equlibration model. 

M.M.M. Meier and R. Wieler provided an overview of the geochemical constraints and motivated new simulations of the giant impact in regimes not yet considered before. A. Reufer and W. Benz performed the necessary simulations and the analysis of the data. All authors discussed the results and implications and commented on the manuscript at all stages. The authors have no competing interests or other interests that might be perceived to influence the results and/or discussion reported in this article.

We thank Maria Schönbächler, Kaveh Pahlevan and Vera Fernandes for discussions and Jay Melosh for providing us with the M-ANEOS package. Andreas Reufer and Matthias M. M. Meier were both supported by the Swiss National Science Foundation. All calculations were performed on the ISIS2 cluster at University of Bern.



\begin{table}
\begin{tabular}{l}
\hspace{-0.8cm} \includegraphics[scale=1.00]{01_paper_tab01}
\end{tabular}
\caption{The constraints set by different isotopic systems on the value of $\delta f_{T}$. Reading example: If the impactor has the same isotopic composition in $\O17$  as Mars ($+0.32 \permil$), the closest-to-impactor value the moon could have still compatible with observations (in this case: $+0.016 \permil$) constrains the maximum depletion of terrestrial material in the moon-forming disk to $5.0\%$. In other words, the disk got only up to $5.0\%$ more material from the impactor than the Earth as a whole (identical to the value in \citep{Wiechert:2001p3543}). It is obvious from this table that the O isotope measurements are the most constraining. For comparison, the canonical model (\emph{cA08}) yields a $\delta f_{T}$ of $-66\%$, incompatible with all listed isotopic systems. Reference keys: [7] \cite{Wiechert:2001p3543}, [10] \cite{2007Natur.447.1102G}, [13] \cite{2000SSRv...92..225S}, [14] \cite{2007Natur.450.1206T}, [25] \cite{2010LPI....41.2653F}, [27] \cite{2009GeCoA..73.5150K}.}
\label{ch05_tbl01}
\end{table}

\begin{figure}
\begin{center}
\includegraphics[scale=2.0]{01_paper_fig01}
\caption{Four snapshots from the $30 \deg$ impact angle and $1.30 \vesc$ impact velocity case (\emph{cC06}) showing cuts through the impact plane. Colour coded is the type and origin of the material. Dark and light blue indicate target and impactor iron; Red and orange show corresponding silicate material. The far right shows the situation at the time of impact. At $0.52 h$, it can be seen how the impactor ploughs deep through the targets mantle and pushes considerable amount of target material into orbit. A spiral arm of material forms and gravitationally collapses into fragments. The outer portions of the arm mainly consist of impactor silicates and escapes, while the fragments further inward enter eccentric orbits. The impactor iron looses angular momentum to the outer parts of the spiral arm and re-impacts the proto-Earth.}
\label{ch05_fig01a}
\end{center}
\end{figure}

\begin{figure}
\begin{center}
\includegraphics[scale=2.0]{01_paper_fig03}
\caption{Comparing post-impact temperatures of the proto-Earth between the grazing reference simulation left (\emph{cA08}) and the head-on case on the right (\emph{cC06}). Colour coded is temperature in K in logged scale. The initial average temperature before the impact inside the target mantle is $\approx 2000 K$.}
\label{ch05_fig01c}
\end{center}
\end{figure}

\begin{figure}
\begin{center}
\includegraphics[scale=2.0]{01_paper_fig02}
\caption{The origin of the disk material highlighted, half a collisional timescale ( $ (\Rimp + \Rtar) / \vimp$ ) after impact. In the grazing reference case (\emph{cA08}), the majority of the proto-lunar disk originates from a spill-over of the impactor. In the head-on cases (\emph{cC01, fB06, iA10}), much more material comes from the target mantle, being pushed out into orbit by the impactor core. Colours are
identical to figure 1. Turquoise on the right shows water ice for the icy impactor case \emph{iA10}.}
\label{ch05_fig01b}
\end{center}
\end{figure}

\includepdf[pages={1-},fitpaper=true]{01_paper_som}

\bibliographystyle{plainnat}
\bibliography{bibliography}


%[1] \citep{1976LPI.....7..120C}
%[2] \citep{Canup:2001p1861}
%[3] \citep{Pahlevan:2007p2065}
%[4] \citep{2011E&PSL.301..433P}
%[5] \citep{Canup:2001p3295}
%[6] \citep{Canup:2004p115}
%[7] \citep{Wiechert:2001p3543}
%[8] \citep{Clayton:1993p3544}
%[9] \citep{2009E&PSL.287...77F}
%[10] \citep{2007Natur.447.1102G}
%[11] \citep{2005E&PSL.236...78W}
%[12] \cite{2000orem.book..197J}
%[13] \citep{2000SSRv...92..225S}
%[14] \citep{2007Natur.450.1206T}
%[15] \citep{2010E&PSL.292..363N}
%[16] \citep{Asphaug:2006p3729}
%[17] \citep{2006Icar..184...39O}
%[18] \citep{1972Thompson}
%[19] \citep{Melosh:2007p3502}
%[20] \citep{Benz:1985p1755}
%[21] \citep{Kokubo:2000p2195}
%[22] \citep{Asphaug:2010p3539}
%[23] \citep{2010Sci...328..884S}
%[24] \citep{2006mess.book..803R}
%[25] \citep{2010LPI....41.2653F}
%[26] \citep{2007Natur.450.1206T}
%[27] \citep{2009GeCoA..73.5150K}


%\citep{Canup:2004p115}  \emph{However, as the impact velocity is increased to 1.1vesc, an increasing amount of escaping material yields lower peak disk masses and angular momenta; in addition, there appears to be a fairly consistent increase in the fraction of disk iron with in- creasing	(vimp /vesc ) for a given impact parameter. Thus we consider (vimp /vesc ) = 1.1 to be an approximate upper limit for a potential lunar-forming impact.}
%maximum disk mass for b = 0.7 to 0.8
%comparable to tillotson runs, less massive disks with more iron, more material outside roche limit
%for larger impact parameters, the impactor tends to re-accumulate, 
%orbital angular momentum most consistent quantities, iron fraction the least consistent
%$\M_{disk} / \Mtot \approx 0.15 \gamma$ 
%disk comes from impactor (more than 70\% for all successful candidates)

%\citep{Canup:2008p3551}
%pre-impact rotation, $\ML = 0.012 \ME, 1 \lem 3.5 \times 10^41 g cm^2 / s$, all previous simulations had $\gamma = 0.11 \dots 0.14$ and $\vimp < 1.1 \vesc$, fast collisions but with a small mass, fast collisions show similarly a decrease in impactor material in disk, still using the same constraint that Linitial = Lfinal, MFe varies widly for high velocities, successful simulations: more than one Moon mass satellite outside Roche limit and $L < 1.2 \lem$, target pre-rotation allow for larger impactors $\gamma = 0.2$, pre-rotating target might be favourable for Moon formation

%grid collisions:
%\cite{Wada:2006p1013}
%problem with low disk resolution (a few percent of the mass are in the disk), gravity dominated process

%\citep{Canup:2010p3713} (SPH vs. CTH)



%accretion:
%\citep{Kokubo:2000p2195}
%disk of 2 to 4 moon masses required (\emph{compact disk}, most of the mass inside the roche limit), mass distribution not important, process becomes more efficient for larger disk angular momentums, two moons are usually on an unstable horse-shoe orbit, evolution time 50 to 1000 kepler times ($T_K 7h$), N-body with collisions
%
%\citep{Ida:1997p3395} scaling law for the moon mass in dependence of the disk mass (10 to 40 \% efficiency, linear to Ldisk)
%
%\cite{Canup:1996p3541} multiple moon formation not a problem, as long as their size sorted on radii. a lunar mass outside the Roche lobe ($2.9 \RE$ for lunar densities).


