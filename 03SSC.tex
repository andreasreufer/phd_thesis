\newpage
\chapter{Similar sized collisions}
\graphicspath{{./03figs/}}


% TODO: show overview plot or collisional orbit

%\section{Required physics}
%compare timescales
%viscous timescale $t = R^2 / \nu$

%\section{interesting quantities}
%\section{accretion efficiency}
%\section{striping efficiency}
%\SSC 
%\textsf{\SSC}


%focus:
%- show parameter space
%- chains from h&r, composition and thermodynamics of such chains
%- deflection angles for h&r, rotation (for h&r)
%- compositional changes for h&r
%- reproduce accr. efficiency in the Earth regime
%- accr. efficiencies for individual components
%- energy partition in bodies (shocks, tidal vs. self-gravity stress)
%- check Kokubo 2010 (critical velocity)
%- check Benz 1999, Angor & Asphaug 2004, Stewart & Leinhardt 2009, Marcus 2009, Marcus 2010
%- analyze ejecta, entropy and energy of ejecta
%- compare central potential of a planetary system with local gravity, compare timescales
%- analyze disk masses & composition


\cite{Benz:1988p3336}
%- 0.1 Mearth target, 1:6 mass ratio, vimp = 2.5-8 vesc, all eroding

\cite{Benz1999Icar..142....5B}
%Benz & Asphaug 1999:
%- MLR / Mtarg ~ Q / Q*
%- parabolic fit

\cite{Canup:2000p3542}
%Canup & Agnor 2000:
%- eccentricities and masses at late stage of terrestrial planet formation
%- critical angular momentum Lcrit
%- satellite impacts are much less common due to 

\cite{Agnor:2004p3329}
%Agnor & Asphaug 2004: accretion efficiency during planetary collisions
%- impact probability of eps: dP = 2*sin(eps)*cos(eps)*deps (Shoemaker 1962)
%- non-disruption doesn't mean merging
%- M1: largest remnant, M2: largest escaping remnant
%- M2 / Mesc < 0.8 for 30deg -> chains
%- two 0.1Mearth mass bodies, chondritic, Tillotson
%- non-accretionary collisions are the norm

\cite{Asphaug:2006p3729}
%Asphaug 2006: h&r planetary collisions
%- 0deg: shocks dominate, 90deg gravity (tides, stresses) dominates
%- "strings of pearls"
%- for large bodies, the impactor, not the target, are destroyed
%- tidal vs. self-gravity stress inversely proportional to mass
%- relativ energy deposition 

\cite{Asphaug:2010p3539}
%Asphaug 2010: SSC collisions and the diversity of planets
%- next-largest bodies (NLB)
%- hit&run common for v_rand = v_inf
%- Safronov number 1-2 in late systems
%- SSC: contact compression timescale  (2r/v_imp) on gravity timescale ( r*sqrt(G*rho) )
%- shear stress exceeds strength above 100km
%- turn-around the target reference frame: the impactor is the altered body
%- Agnor 1999: angular momentum easily above dynamical stability
%- SSC scale-invariant to the first order
%- 45deg as the median impact angle (Shoemaker 1962)
%- small r/R -> impact cratering
%- "grazing": center of impactor skims tangential to the target
%- impact cratering: angle gives all or nothing, SSC is a continuum
%- for r_core = 0.5*r -> 30deg to 90deg miss each others cores
%- h&r prevalent for NLB
%- NLBs get mantle-stripped
%- tidal disruption important for small bodies
%- icy collision might be similar to rocky ones (ice : rock ~ rock : iron)
%- iron-enriched fragments from chain-events
%- h&r happen to about half the NLB?


\cite{2009ApJ...700L.118M}
Marcus 2009:
%- confirmation of Stewart & Leinhardt 2009 for bodies > 100km
%- masses 1., 5., 10. Mearth, ratios 0.25, 0.50, 0.75
%- scaling relationship for MFe / Mlr
%- compositional changes require: a) small impact angle or b) small & fast impactor

\cite{2010ApJ...712L..73M}
%Marcus et al. 2010: Minimum radii from Super-Earths: Constraints from GIs
%- super-Mercuries are not expected, as striping bodies would have to be > 10 MEarth

\cite{Stewart:2009p3265}
%Stewart & Leinhardt 2009:
%- catastrophic disruption criteria
%- Mlr / Mtarg replaced by Mlr / Mtot, Q* sclaed to reduced mass in kinetic energy

\cite{2010ApJ...714L..21K}
%Kokubo & Genda 2010:
%- about half the collisions do not accrete
%- orbit integration of 16 bodies, total mass 2.3MEarth
%- mass ratios: 1.0, 0.66, 0.50, 0.33, 0.25, 0.17, 0.11, Mtot = 0.2-2.0MEarth
%- impact velocities: 1.0-3.0 v_esc, angles: 0-75deg
%- criteria for critical impact velocity v_cr / v_esc for which accretion occurs
%- critical spin angular velocity

\cite{2011arXiv1105.4616E}
%Elser et al, 2011:
%- Moon formation in planetary systems

\bibliographystyle{plainnat}
\bibliography{bibliography}



