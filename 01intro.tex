\cleardoublepage
\chapter{Introduction}
\label{ch01}

spurred interest in extrasolar planets, new instruments will deliver data about terrestrial planets

brief planet formation theory overview: physical processes 
midplane accumulation in $10^4$ years

dust -> grains (sticking, molecular forces dominate) up to cm sized bodies
meter sized barrier, gravity is too weak, but velocities become large (gas drag becomes small, velocities go to keplerian speeds)
\cite{Benz1999Icar..142....5B}

above the meter barrier, gravitational focussing let's the largest bodies grow the fastest (runaway growth)


olichargic growth, when largest embroys becomes 100 larger than typical planetesimal mass. embryo dynamics determines now the \cite{1993Icar..106..210I} \cite{2010ApJ...714L.103O}
embryos empty their feeding zones


dynamical friction stabilizes large bodies, and excites small bodies, when no more planetesimals are around, bodies become excited again and start to collide again

radial positions change, large radial mixing, similar-sized collisions \cite{Chambers:2004p4098} \cite{Chambers:2001p2105}

mixing beyond ice line 



vortex or mri? \cite{Johansen:2007p37} turbulence in general \citep{2008ApJ...686.1292I} planetesimals form
oligarchic growth

gas is removed (photoevaporation due to stars UV)

giant planet formation due to core accretion \citep{1996Icar..124...62P}p

why are collisions important?
late stages: single, large events determine:

the final state of volatile depletion \citep{2001E&PSL.192..545H}
triggering of core formation \cite{1992Icar..100..326T}
change of primordial atmospheres \cite{2002DPS....34.2804A}

give direct examples:
moon -> chapter 5 \cite{Canup:2001p1861} \cite{1987Icar...71...30B} \cite{1975Icar...24..504H} \cite{1976LPI.....7..120C}
mercury \cite{Benz:1988p3336}
pluto-charon \cite{Canup:2005p1987}

statistical results for evolution codes, to understand planet formation in general

especially terrestrial planet formation \cite{2006Icar..184...39O} 

application of impacts mars -> chapter 4


\cite{Chambers:2004p4098}

terrestrial planet formation: important as core formation for giants but also for planets themselves


what do the chapters have in common?

compositional changes from collisions

method side: SPH as a robust method for collisions, thermodynamics are important

SPH has been used for single cases only

progress in computational methods and computing power allows massively parallel 



%\section{Motivation}
%collisions as event between two bodies
%cratering to similar sized collisions
%importance in solar system formation
%outcomes give clues 
%solar system dimensions
% refer to chapters: 

% motivation for simulating collisions
% introduce collision basics and terms
% give examples of collisions

% check Augustins Diss

%\section{Collision basics}
%introduce terms: impactor, target, impact velocity, impact angle

\bibliographystyle{plainnat}
\bibliography{bibliography}
