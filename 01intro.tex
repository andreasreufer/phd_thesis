%\cleardoublepage

%\newpage
%\thispagestyle{empty}
%\mbox{}

\begin{figure}
\begin{center}
\includegraphics[scale=0.7]{01figs/01planet-collision.jpg}
\caption{Artists impression of the Moon forming giant impact. Image courtesy of NASA/JPL-Caltech}
\label{ch01_fig01}
\end{center}
\end{figure}


\chapter{Introduction}
\label{ch01}
\graphicspath{{./01figs/}}
Collisions between bodies of various sizes are an important process during all stages of planet formation. A collision consists of two macroscopic bodies, the larger body termed the \emph{target} and the smaller body the \emph{impactor} approaching each other and resulting in a collisional outcome depending on various parameters. When the target is considerably larger than the impactor, a collision is also often called an impact. Collisions occur at various stages during planet formation, with mass and size ranges of the involved bodies varying several tens of magnitudes from dust grains up to Jupiter sized planets. This thesis covers different types of collisions all modeled with the same type of method described in detail in chapter \ref{ch02}. The following brief overview of planet formation helps in putting these different types of collisions in context.

The field of planet formation lead a neglected existence besides related fields like stellar formation or cosmology for a long time. While telescopes allowed mankind to observe stars beyond the solar system for several centuries, the knowledge about planets has long been limited to the ones in our own system. The relative small luminosity of planets compared to stars makes it even today almost impossible to directly detect planets even for stars in our neighborhood. Without any further known systems, the question wether planet formation is a common process alongside of stellar formation remained non answerable question. But the common wisdom was, that the solar system is a typical result of planet formation and that hypothetical extrasolar systems should look similarly. This is no surprise, as planet formation models primarily aimed at explaining the solar system.

The discovery of the first extrasolar planet \emph{51 Pegasi b} around a solar-type star by \cite{1995Natur.378..355M} caused an overnight revolution in the field. 
Before this discovery, nobody deemed it possible that a Jupiter-mass planet could exist as close as $0.05 AU$ to its host star. As of July 2011, ongoing ground- and space-based observations have discovered more than 560 extrasolar planets. All of them are relatively massive and close-in, more resembling the originally observed \emph{51 Pegasi b} than the planets in our own solar system. This is primarily an effect of observational bias, favoring the detection of massive, close-in planets. Nevertheless, a model of planet formation also has to explain those systems.

According to the current understanding of planet formation, the formation process of a planetary system can be divided into different stages: A part of a giant molecular cloud, composed of gas (mainly H \& He) and dust, collapses and starts to form a rotating disk due to the conservation of angular momentum. When density and pressure are large enough in the center, hydrogen burning starts and a new star is born. What resembles more a rotating lump of gas at first, changes over time into a disk with decreasing pressure, density and temperature as a function of the distance to the star. This happens over a timescale of $\approx 10^5$ years.


Due to the radial pressure gradient, the gas in the disk rotates with a sub-keplerian velocity around the star. In the vertical direction of the disk, there is also a positive pressure gradient towards the midplane. While the dust is tightly coupled to the rotation of the gas around the star, it is pressureless in itself and therefore settles towards the disks midplane over a timescale of $10^4$ years. The $\mu m$-sized dust starts to coagulate due to strong van-der-Waals forces into larger grains \citep{2010A&A...513A..56G}. As the grains grow towards bigger planetesimals their relative strength decreases as inter-molecular forces increase $\sim r^2$ while intertial forces increase with $\sim r^3$. With the diameter approaching the \emph{meter-sized barrier}, further growth of the solid bodies is stopped \citep{Benz1999Icar..142....5B}.

How this barrier can be overcome is still unknown. Recent suggestions have been the trapping of dust in vortices, either caused by magneto-rotational-instability \citep{Johansen:2007p37} or by turbulence in general \citep{2008ApJ...686.1292I}. In those scenarios, the trapped dust instantly collapses under its own self-gravity into \emph{Ceres} sized objects ($\approx 10^{-4} \ME$). In this regime, gravity becomes strong enough to hold together those objects and further mass is accreted. Due to gravitational focussing, the zone where those bodies accreted mass from increases along with their mass and leads to \emph{runaway accretion}. During this stage, the amout of gas in the disk already has decreased, either by accretion onto the central star or due to photo-evaporation due to the young stars UV radiation. After $\approx 10$ million years, the gas has largely disappeared.

The growing planetesimals now turn into planetary embryos and become a victim of their own success: When an embryo reaches a mass roughly $ 100\times$ larger than the typical planetesimal mass in its vicinity, this feeding zone is being emptied and the accretion rate goes down again. Neighboring embryos can catch up and the \emph{oligarchic} accretion regime kicks in \citep{1993Icar..106..210I, 2010ApJ...714L.103O}.

During oligarchic growth, the individual embryos obtain similar masses and are orderly spaced according to their feeding-zones. The remaining small planetesimals stabilize the orbits of the embryos through dynamical friction, while their own eccentricities and inclinations are pumped up. This leads to a slow removal of the small planetesimals due to accretion onto embryos, loss into the central star or due to ejection. Without the dampening by the small planetesimals and the gas vanished, the embryos orbits become unstable again. Embryos eventually get on crossing orbits. The resulting two-body interactions chaotically re-arrange the embryos in the disk and also lead to collisions of embryos \cite{Chambers:2001p2105, Chambers:2004p4098}. With all embryos in the same mass regime, these collisions occur between similar-sized bodies and lead to various, non-trivial outcomes \citep{Asphaug:2010p3539}. Such collisions and their outcomes are discussed in detail in chapter \ref{ch03}. Work on modeling the stage between the formation of embryos and their final accretion into proto-planets \citep{Chambers:2001p2105, 2006Icar..184...39O} often makes simplified assumptions about these similar-sized collisions. The data provided in chapter \ref{ch03} permits such models to incorporate a more realistic collision handling and to model this stage of planet formation more accurately.

Similar-sized collisions alter bodies in a profound way: Properties like composition, bulk density \citep{Benz:1988p3336} or the rotation can be changed drastically by such collisions. They are also believed to have trigger core formation \citep{1992Icar..100..326T}. Consequently, the final properties of terrestrial planets are strongly influenced by the last massive collision they underwent: Primordial atmospheres can be removed \citep{2002DPS....34.2804A} or the planet can be depleted of its volatiles \citep{2001E&PSL.192..545H}. Another interesting effect of such collisions is satellite formation. Along with the Pluto-Charon binary system \citep{Canup:2005p1987}, the Earth-Moon system is most probably a direct outcome of a late giant collision between the proto-Earth and a roughly Mars sized body \citep{1975Icar...24..504H, 1976LPI.....7..120C, 1987Icar...71...30B, Canup:2001p1861}. In chapter \ref{ch05}, we look into an alternative impact scenario explaining the Moons formation.

With the late heavy bombardment marking the last major event in the solar systems history \citep{2005Natur.435..466G}, planet formation as a process has finished $\approx 4$ billion years ago in our planetary system. Nevertheless impacts of small remnants (for example from the asteroid or the Kuiper) belt onto planets and their moons still take place. While not delivering substantial mass relative to the target body, the craters such impacts leave behind give important clues about the rheology and age of planetary surfaces. Chapter \ref{ch04} presents modeling work on martian craters which have been observed only months after their formation \cite{2009Sci...325.1674B}. 

With the ongoing improvements in the detection capabilities for extrasolar planets, observations in the next decades will deliver information about smaller mass objects in the terrestrial planet regime. This data will put terrestrial planet formation models up to a test, similarly as the discovered giant exoplanets have challenged giant planet formation models. The next few decades will surely be an exciting time for the field of planet formation.

%statistical results for evolution codes, to understand planet formation in general 
% method side: SPH as a robust method for collisions, thermodynamics are important
% SPH has been used for single, large cases mainly \cite{2005Natur.435..629S}
% progress in computational methods and computing power allows massively parallel 

%\section{Motivation}
%collisions as event between two bodies
%cratering to similar sized collisions
%importance in solar system formation
%outcomes give clues 
%solar system dimensions
% refer to chapters: 

% motivation for simulating collisions
% introduce collision basics and terms
% give examples of collisions

% check Augustins Diss

%\section{Collision basics}
%introduce terms: impactor, target, impact velocity, impact angle

\bibliographystyle{apalike}
\bibliography{bibliography}
