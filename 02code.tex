\graphicspath{{./02figs/}}

\chapter{Methods}
%\includepdf[pages={2-},fitpaper=true]{/tree_musings}


\section{Gravity}
\subsection{N-Body problem}

\subsection{Barnes and Hut Tree}


\section{Fluid- and Thermodynamics}
\subsection{Theory}

\subsection{SPH formalism}
advantages and disadvantages of SPH
formalism

\subsection{SPH implementation used}
standard SPH vs. integrated density, vs.\\
deriving SPH sums from Navier-Stokes\\
surface tension issues, miscible SPH, show m1H vs. m3H\\
refer to price\\

\subsection{Time integration}


\subsection{ANEOS}
tables, interpolating, iterating, issues\\
show isobars\\
entropy vs. energy integration\\

\section{Implementation}
\subsection{Organizing particles in Trees}
building a tree\\
deleting a tree\\
neighbour search\\
tree walks\\
accelerating a tree: show skip, next and parent pointers by simple tree examples\\

\begin{figure}[htbp]
\begin{center}
\includegraphics[scale=0.6]{cell_wiring.pdf}
\caption{Wiring scheme of an octree cell node a depth $n$ with particle childs in subvolume 1 and 6 and a cell node as child 3. Parent pointers allow going upwards in the tree, child pointers downwards. Following the next pointers results in a pre-order tree traversal, where taking the \it{skip} pointer skips a cells subtree. Note that a \it{next} pointer }
\label{fig02walks}
\end{center}
\end{figure}


\begin{figure}[htbp]
\begin{center}
\includegraphics[scale=0.6]{orderwalks.pdf}
\caption{Post-order vs. pre-order recursors}
\label{fig02walks}
\end{center}
\end{figure}


\subsection{Tree parallelization: shared vs. distributed memory}
intro: shared vs. distributed memory\\

SPHLATCH v1\\
parallelizing a particle code: distributed sums vs. ghosts approach\\
parallelizing a tree\\
show ghost approach\\

SPHLATCH v2\\
shared memory approach\\
load balancing approach, how this could be memory-\\
show cost (gravity, NS, total) for giant impact\\
caching issues\\

building the tree:

\section{Algorithms}
clump detection\\
FOF vs. potential, not parallelized\\
initial conditions: show theoretical vs. unevolved vs. evolved densities, show example of impact with standard SPH and miscible SPH (chondritic, use moon case), setting up a \SSC \\


\subsubsection{Neighbour search}

\subsection{setting up initial conditions}
use of spheres for simplicity reason

\citep{Barnes:1986p2853}
\citep{Monaghan:2005p2677}
\citep{Price:2004p2613}

\bibliographystyle{plainnat}
\bibliography{bibliography}


