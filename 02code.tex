\graphicspath{{./02figs/}}

%\tableofcontents

\chapter{Methods}
Collision events are characterised by a 
range of dynamical 

%\subsection*{Introduction}
%\includepdf[pages={2-},fitpaper=true]{/tree_musings}

Motivation to use SPH, spatial dynamics of collisions, SPH particles are where the mass is (not entirely true for collisions, but necessary), free boundaries



Lagrangian nature vs. grid techniques, complexity of AMR vs. simplicity of SPH

Neighbour search is the hardest thing about SPH



In this chapter we will discuss, how collisions in the strengthless regime as appearing in planetary systems can be modelled.
The first section gives a short review the governing physics of such collisions and will provide us with a closed set of equations, namely the Navier-Stokes equations and the 
This set is far too complicated to be solved in an analytical way. 
Smoothed particle Hydrodynamics (SPH) is a method which discretises fluids with particles of finite volume and presents a framework to solve the equations describing the fluid.
The second section derives different SPH formulations of the Navier-Stokes equations and demonstrates the advantages and disadvantages of each formulation.

gravity section: 

The initial conditions and also the outcome of collisional events can be charaterized by a 

Smoothed particle hydrodynamics (SPH) is a numerical method which can discretize 

(insert a nice picture of equations, code lines and numbers here)

\newpage

\section{Fluid- and Thermodynamics}
A continuum model 

%%
%   F L U I D S
%%
\subsection{Theory of Fluids}

% TODO
\begin{equation}
\label{ch02_fld01_eq001}
\frac{\partial \rho}{\partial t} + \rho \nabla \mathbf{v} = 0
\end{equation}

\begin{equation}
\label{ch02_fld01_eq002}
\rho \frac{\partial \mathbf{v}}{\partial t} + \rho ( \mathbf{v} \nabla ) \mathbf{v} = - \nabla p + \eta \Delta \mathbf{v} + ( \zeta + \eta ) \nabla ( \nabla \mathbf{v} )
\end{equation}

\label{ch02_fld01_eq003}
\begin{equation}
- \frac{\partial \mathbf{v}}{\partial t} =  ( \mathbf{v} \nabla ) \mathbf{v} + \frac{\nabla p}{\rho}
\end{equation}

The Lagrangian of a inviscous fluid is given by
\begin{equation}
\label{ch02_fld01_eq004}
\Lagr = \int \Big( \frac{1}{2} \rho \vv ^2 - \rho u \Big) dV
\end{equation}

\citep{Thorne:2008}
\citep{2004Nyffeler}
%%
%   S P H
%%
\subsection{SPH formalism}
The basic idea of SPH is to interpolate all continuum variables in space with an interpolant.
If our variable is $A(\rv)$ defined at all points in space $\rv$, we start with the basic equality

\begin{equation}
\label{ch02_sph01_eq001}
A(\rv) = \int  \delta(\rv - \rvs) A(\rvs) \ud \rvs
\end{equation}

We now replace the $\delta$-function with a smoothing kernel $W$, which has typical width $h$, the smoothing length, and which fulfils the following two criteria

\begin{equation}
\label{ch02_sph01_eq002}
\limhzero W(\rv - \rvs, h) = \delta(\rv - \rvs) \hspace{1.0cm}
\int W(\rv - \rvs, h) \ud \rvs = 1
\end{equation}

Equation \ref{ch02_sph01_eq001} can now be re-written as 

\begin{equation}
\label{ch02_sph01_eq003}
A(\rv) = \int  W(\rv - \rvs, h) A(\rvs) \ud \rvs + O(h^2)
\end{equation}

The variable $A(\rv)$ is now smoothed out over the characteristic scale $h$. By replacing the $delta$-function with a kernel of finite length, an error in the order of $h^2$ is introduced. Now comes the particle part: The continuum is replaced by a finite set of points (\emph{particles}), on which the quantities are defined. The integral can now be replaced by a sum with a finite number of summands:

\begin{equation}
\label{ch02_sph01_eq004}
A(\rv) \approx \sum_{j} W(\rv - \rvj) A_i V_{j} \hspace{1.0cm} V_{j} = \frac{m_j}{\rho_j}
\end{equation}

$V_{j}$ is the particle volume and is usually given assigning each particle a mass and dividing it by the density. The particles used in the summation are called \emph{neighbours} and are indexed here with $j$. For most purposes, the variable only needs to be known at particle position $i$. For convenience, we write the kernel used between two particles as $\Wij = W( \rvi - \rvj, h)$

If the smoothing length goes to zero and the number of neighbours goes to infinity, the sum \ref{ch02_sph01_eq004} approximates the integral \ref{ch02_sph01_eq001} exactly. This is the \emph{SPH limit}.

Spatial derivatives are easily given by taking the partial derivative of the integral interpolant variable and replacing it again by the sum over all neighbours:

\begin{equation}
\label{ch02_sph01_eq005}
\nabla A(\rv_i) = \frac{\partial}{\partial \rv}\int A(\rvs) \ud \rvs \frac{\partial}{\partial \rv}\ W(\rv_i - \rvs, h) + O(h^2)
\approx \sum_{j} A_j V_{j} \nabla \Wij 
\end{equation}

Rotations of variables can be similarly obtained:
\begin{equation}
\label{ch02_sph01_eq006}
\nabla \times A_i \approx \sum_{j} A_i V_{j} (\nabla \times \Wij)
\end{equation}

Like in in grid based schemes, there are different ways of calculating spatial derivatives in SPH.
Depending on the actual application of the derivative, whether it is to get the velocity divergence or a pressure gradient, there are better suited variants which show better error behaviour. They will be presented shortly.

The kernel has to fulfill the two constraints posed by equations \ref{ch02_sph01_eq002}. A natural choice would be a normalized Gaussian, but this has the unpleasant of never vanishing. All particles, also distant ones, would contribute to every individual SPH sum, making the method very inefficient. A better choice are therefore functions with compact support: Functions which vanish at sufficiently large distances. A common selection are cubic splines

\begin{equation}
\label{ch02_sph01_eq007}
W(| \rv |, h) = \frac{\sigma}{h^\nu \pi} \left \{ \begin{array}{lll}
1 - \frac{3}{2}q^2 + \frac{3}{4}q^3 & r \leq h & \\
\frac{1}{4}(2-q)^3 & h < r  \leq 2h & \hspace{1.0cm} q = r / h\\
0 & 2h < r & \\
\end{array} \right. 
\end{equation}

where $\nu$ stands for the number of spatial dimensions and $\sigma = \big( \frac{2}{3}, \frac{10}{7\pi}, \frac{1}{\pi} \big) $ for $\big( 1,2,3\big)$ dimensions. The Kernel vanishes for distances greater than $2h$, reducing SPH sums to contributing terms of only the nearest neighbours. The first derivative is given by 

\begin{equation}
\label{ch02_sph01_eq008}
\nabla W(\rv, h) = \frac{\rv}{r} \frac{\sigma}{h^\nu \pi} \left \{ \begin{array}{ll}
1 - 3q + \frac{9}{4}q^2 & r \leq h \\
\frac{3}{4}(2-q)^2 & h < r  \leq 2h \\
0 & 2h < r \\
\end{array} \right. 
\end{equation}

and is also continuous. 

The choice of the smoothing length is a trade-off. It is large compared compared to the spacing between the particles, then the variables are smoothed out over too large distances and small features cannot be resolved. If the smoothing length is too small, the variables are not sufficiently smoothed and get very noise, ultimately leading to instabilities. As a golden rule, the smoothing length should be chosen, such as the number of neighbors is 50 inside a radius of $2h$. For a hexagonal HCP lattice in 3D with a lattice constant $k$, this leads to $h \approx 0.85 k$. 

\subsection{standard SPH}
The original and most widely used version of SPH uses the density and particle mass to calculate the particle volumes ($V_j = m_j / \rho_j)$ and the density to interpolate all variables. The density is given by the simple particle sum

\begin{equation}
\label{ch02_sph02_eq001}
\rho_i = \sum_{j} m_j \Wij
\end{equation}

The velocity divergence $\nabla \vv$ could in principle be written as
\begin{equation}
\label{ch02_sph02_eq002}
\nabla \vv_i = \sum_{j} \frac{m_j}{\rho_j} \vv_j \nabla \Wij
\end{equation}

but this form has the disadvantage of being asymmetric and will ultimately lead to a non-conservative form for the linear momentum. It is better to apply the \emph{second golden rule} of SPH \citep{Monaghan:1992p3721} to re-write the velocity divergence term with the density placed inside the operators:

\begin{equation}
\label{ch02_sph02_eq003}
\nabla \vv = \frac{1}{\rho} \Big( \nabla (\rho \vv) - \vv \nabla \rho \Big)
\end{equation}

so that the corresponding SPH sum becomes 

\begin{equation}
\label{ch02_sph02_eq003}
\nabla \vv_i = \frac{1}{\rho_i} \sum_{j} \frac{m_j}{\rho_j} \rho_j \vv_j \dWij - \frac{1}{\rho_i} \vv_i \sum_{j} \frac{m_j}{\rho_j} \rho_j \dWij = \frac{1}{\rho_i} \sum_{j} m_j \vvij \dWij 
\end{equation}

with $\vvij = \vv_j - \vv_i$, which is a symmetric form for the velocity divergence. If we take the partial time derivative of \ref{ch02_sph02_eq001}, where only the particle positions in the Kernel depend explicitly on time, we get: 

\begin{equation}
\label{ch02_sph02_eq004}
\frac{\partial \rho_i}{\partial t}  = \sum_{j} m_j \Big( \frac{\partial}{\partial t} \rvij \Big) \nabla \Wij = \sum_{j} m_j \vvij \nabla \Wij 
\end{equation}

So this equation together with the velocity divergence equation \ref{ch02_sph02_eq003} fulfills the continuity equation \ref{ch02_fld01_eq001} for each particle.

The equations of motion can be derived in two different ways: With the inviscous equation of motion of the Navier-Stokes equation \ref{ch02_fld01_eq003}, a symmetric formulation for the pressure gradient term $\nabla p / \rho$ can be found with the trick used above for the velocity divergence. 
A more elegant way is to derive the equation of motion directly from the Lagrangian given by equation \ref{ch02_fld01_eq004}, whose SPH formulation is

\begin{equation}
\label{ch02_sph02_eq004}
\Lagr = \sum_{j} m_j \Big( \frac{1}{2} \vv_j^2 - u(\rho_j, s_j) \Big) 
\end{equation}

where $u$ is the specific internal energy of a fluid particle dependent only on its density and specific entropy $s$. The Euler-Lagrange equations for a single particle $i$ is

\begin{equation}
\label{ch02_sph02_eq005}
\frac{d}{dt} \Big( \frac{\partial \Lagr}{\partial \vv_i} \Big) - \frac{\partial \Lagr}{\partial \rv_i} = 0
\end{equation}

which yields 

\begin{equation}
\label{ch02_sph02_eq006}
m_i \frac{d \vv_i }{dt} = \sum_{j} m_j \frac{\partial u_j}{\partial \rho_j} \frac{\partial \rho_j}{\partial \rv_i}
\end{equation}

for constant entropy, which is the case in the absence of shocks. The first law of thermodynamics for a particle is given by

\begin{equation}
\label{ch02_sph02_eq007}
\frac{u_j}{\rho_j} = \frac{p_j}{\rho_j^2}
\end{equation}

and the divergence of the density is directly given by

\begin{equation}
\label{ch02_sph02_eq008}
\frac{\partial }{\partial \rv_i} \rho_j
= \sum_{k} m_k \frac{\partial}{\rv_i} W_{jk} 
= \sum_{k} m_k \nabla W_{jk} ( \delta_{ji} - \delta_{ki})
\end{equation}

where the kernel vanishes except if $i = j$ and if $i = k$. So equation \ref{ch02_sph02_eq006} yields

\begin{equation}
\label{ch02_sph02_eq009}
m_i \frac{d \vv_i }{dt} = \sum_{j} m_j \frac{p_j}{\rho_j^2} \sum_{k} \nabla_i W_{jk} ( \delta_{ji} - \delta_{ki})\\
 = m_i \sum_{j} m_j \Big( \frac{p_i}{\rho_i^2} + \frac{p_j}{\rho_j^2} \Big) \nabla W_{ij}
\end{equation}

and leaves us with a symmetric SPH sum for the acceleration of a particle in absence of shocks:
\begin{equation}
\label{ch02_sph02_eq010}
\frac{d \vv_i }{dt} = \sum_{j} m_j \Big( \frac{p_i}{\rho_i^2} + \frac{p_j}{\rho_j^2} \Big) \nabla W_{ij}
\end{equation}

The derivative of the kernel is anti-symmetric ($\nabla W_{ij} = - \nabla W_{jj}$), therefore each interacting particle pair fulfils Newtons third law \emph{actio-reactio} and linear momentum is conserved. Angular momentum is also conserved, which can be shown by taking the time derivative of the $\frac{d}{dt} \sum_i (m_i \rv_i \times \vv_i)$ where some index magic \citep{Price:2004p2613} again shows that all terms cancel each other out in the sum over all particles.

In most cases, the internal enery of the fluid is non-constant, so we also need a equation for the change in internal energy due to compressional heating.
The rate of change in specific internal energy is given by 

\begin{equation}
\label{ch02_sph02_eq011}
\frac{du}{dt} = -\frac{p}{\rho} \nabla \vv
\end{equation}

Using the velocity divergence SPH sum \ref{ch02_sph02_eq003}  yields

\begin{equation}
\label{ch02_sph02_eq012}
\frac{du_i}{dt} = \frac{p_i}{\rho_i^2} \sum_{j} m_j \vvij  \dWij
\end{equation}

An SPH sum of this equation consistent with the equation of motion from above can be obtained by applying the golden rule again and putting the density inside the divergence operators:

\begin{equation}
\label{ch02_sph02_eq013}
\frac{du}{dt} = - \nabla \Big( \frac{p \vv}{\rho} \Big) + \vv \nabla \Big( \frac{p}{\rho} \Big)
\end{equation}

The corresponding SPH sum yields the asymmetric equation

\begin{equation}
\label{ch02_sph02_eq014}
\frac{du_i}{dt} = - \sum_{j} m_{j} \frac{p_j \vv_j}{\rho_j} \dWij + \vv_i \sum_{j} m_{j} \frac{p_j}{\rho_j^2}  \dWij = \sum_{j} m_j \vvij \frac{p_j}{\rho_j^2}  \dWij
\end{equation}

A symmetric formulation can be gained by taking the average of equations \ref{ch02_sph02_eq012} and \ref{ch02_sph02_eq014}:

\begin{equation}
\label{ch02_sph02_eq015}
\frac{du_i}{dt} = \frac{1}{2} \sum_{j} m_{j} \vvij \Big( \frac{p_i}{\rho_i^2}  + \frac{p_j}{\rho_j^2} \Big) \dWij
\end{equation}

\subsection{Resolving shocks}
Shocks are changes of the characteristics of a fluid on the spatial scale of the molecules mean-free path. Discontinuities arise in the fluid approximation, usually in density, pressure and temperature.
Numerical schemes tend to react to these discontinuities with unphysical oscillations behind the shock front, because the sharp changes cannot be resolved properly.
Several approaches exist to tackle this problem. Godunov-schemes solve the Riemann problems on both sides of the shock exactly between computational elements and in a second step superpose all pair-wise solutions for each element. In case of SPH this can be done by solving the Riemann problem for each particle against its neighbours \citep{Monaghan:1997p3938}.
Another approach is the von Neumann and Richtmyer viscosity: By introducing a small amount of viscosity, the \emph{artificial viscosity}, the shock front is smoothed out and the oscillations disappear. This can be implemented in SPH by adding additional terms to the momentum and energy equation. The most common used variant is that given by \citep{Monaghan:1992ARAA..30..543M}:

\begin{equation}
\label{ch02_sph02_eq016}
\frac{d \vv_i }{dt} \Big|_{AV} = \sum_{j} m_j \frac{-\alpha c_{ij} \mu_{ij} + \beta \mu_{ij}^2}{\rho_{ij}} \dWij \hspace{1.0cm} \mu_{ij} = \frac{h \vvij \rvij}{\rvij^2 + 0.01h^2}  \hspace{0.5cm} \mathrm{if}  \hspace{0.5cm} \rvij \vvij < 0
\end{equation}

where $\rho_{ij}$ is the averaged density between two particles and $\mu_{ij}$ is a signal velocity with an order of magnitude of the speed of sound. The artificial viscosity term is only applied in situations of compression, where $\rvij \vvij < 0$. The choice of constants is usually $\alpha = 1$ and $\beta = 2 \alpha$. The $\beta$-term is the actual von Neumann and Richtmyer term and becomes dominant in strong shocks where the velocity difference $\vvij$ between two particles in the shock front becomes large. Momentum conservation is still conserved by the symmetric form of the artificial viscosity term.

The contribution of artificial viscosity to the internal energy can be found by requiring that artificial viscosity itself should be energy conserving. The specific energy change due to the dissipative is given by

\begin{equation}
\label{ch02_sph02_eq017}
\frac{de_i}{dt} \Big|_{AV}  = \frac{du_i}{dt}  \Big|_{AV} + \vv_i \frac{d\vv_i}{dt}  \Big|_{AV} = 0
\end{equation}

and should be zero. This leads to the thermal contribution

\begin{equation}
\label{ch02_sph02_eq018}
\frac{du_i }{dt} \Big|_{AV} = - \sum_{j} m_j \vvij \frac{-\alpha c_{ij} \mu_{ij} + \beta \mu_{ij}^2}{\rho_{ij}} \dWij
\end{equation}

with the same choice for the signal velocity $\mu_{ij}$ and numerical constants as in equation \ref{ch02_sph02_eq016}. 

A disadvantage of this approach is that this viscosity also acts outside of shocks in situations of weak compression or shear flows. There exist several approaches to fix this issue by detecting shocks more cleverly \citep{Morris1997J.-Comput.-Phys.Morris} than the simply checking for compression or recognizing shear flows explicitly \citep{Balsara1995JCoPh.121..357B} and suppressing the artificial viscosity in situations where it is not needed. For the simulation of collisions, this is not a big problem, therefore we don't include such fixes.

\subsection{variable smoothing length}
In order to meet the requirement of $50$ neighbors per particle, the smoothing is adjusted for each particle individually. For constant particle masses the density is proportional to the particle density $\delta_i$, which again is inverse proportional to the particle volume and therefore to the inverse third power of the smoothing length in 3D:

\begin{equation}
\label{ch02_sph02_eq026}
\rho_i \sim \delta_i \sim \frac{1}{V_i} \sim \frac{1}{h_i^3}
\end{equation}

So the rate of change of the smoothing length is given by using the continuity equation \ref{ch02_sph02_eq005} yields

\begin{equation}
\label{ch02_sph02_eq027}
\frac{d h_i}{dt} = - \frac{h_i}{3 \rho_i} \frac{d \rho_i}{dt} = \frac{1}{3} h_i \nabla \vv_i
\end{equation}

This keeps the number of neighbours exactly constant, if the particle density changes isotropically. Usually this is not the case and the number of neighbors ($N_N$) becomes either too large or too small. In such cases it is useful to overcorrect the smoothing length with the global maximum of velocity divergence ($\nabla \vv_{max} = \max_{i} \nabla \vv_i$):

\begin{equation}
\label{ch02_sph01_eq028}
\frac{d h_i}{d t} = h_i ( k_1 \nabla \vv_{max}+ k_2 \frac{1}{3} \nabla\vv_i - k_3 \nabla \vv_{max} )
\end{equation}

\begin{equation}
\label{ch02_sph01_eq029}
\begin{array}{ll}
k_1 = \frac{1}{2} (1 + \tanh{- \frac{N_N - N_{N,min}}{5}} ) & N_{N,min} = \frac{2}{3} N_{N,opt} \\
k_2 = 1 - k_1 - k_3 & \\
k_3 = \frac{1}{2} (1 + \tanh{\frac{N_N - N_{N,max}}{5}} ) & N_{N,max} = \frac{5}{3} N_{N,opt}\\
\end{array}
\end{equation}

If the number of neighbours is within $\frac{2}{3} N_N$ and $\frac{5}{3} N_N$, the expression \ref{ch02_sph02_eq027} is dominant. If it's below or above the optimal number, the change rate of the smooting length is over- or under-corrected with the maximum velocity divergence. In practice, this scheme keeps the number of neighbors in reasonable limits, without introducing any instabilities.

\subsection{Integration}
Together with an equation of state which gives us the pressure $p_i(u_i, \rho_i)$ and as a side effect also the speed of sound $c_i(u_i, \rho_i)$, we have a closed set of equations describing the fluid which can be integrated in time. A common choice due to its simplicity and low number of derivation steps required is the predictor-corrector scheme \citep{Press2002nrc..book.....P}. For the integration of particle positions we use the second-order form, which yields the prediction step

\begin{eqnarray}
\label{ch02_sph02_eq019}
\rv_i^{(p)} = \rv_i^{(0)} + \frac{1}{2} ( 3 \vv_i^{(0)} - \vv_i^{(-1)} )\Delta t \\
\vv_i^{(p)} = \vv_i^{(0)} + \frac{1}{2} ( 3 \av_i^{(0)} - \av_i^{(-1)} )\Delta t
\end{eqnarray}
and the correction step

\begin{eqnarray}
\label{ch02_sph02_eq020}
\rv_i^{(1)} = \rv_i^{(0)} + \frac{1}{2} ( \vv_i^{(p)} - \vv_i^{(0)} )\Delta t \\
\vv_i^{(1)} = \vv_i^{(0)} + \frac{1}{2} ( \av_i^{(p)} - \av_i^{(0)} )\Delta t
\end{eqnarray}

The integrator requires the evaluation of the derivative $\av_i( \rv_i, \vv_i)$ two times. As the predicted variables $(\rv_i^{(p)}, \vv_i^{(p)})$ and previous variables $(\rv_i^{(-1)}, \vv_i^{(-1)})$ are never used at the same time, it is sufficient to store only one additional set of variables.

Variables defined by first-order differential equations like the internal energy, use the first-order prediction step

\begin{equation}
\label{ch02_sph02_eq021}
u_i^{(p)} = u_i^{(0)} + \frac{1}{2} ( 3 \dot{u}_i^{(0)} - \dot{u}_i^{(-1)} )\Delta t \\
\end{equation}
and the correction step

\begin{equation}
\label{ch02_sph02_eq022}
u_i^{(1)} = u_i^{(0)} + \frac{1}{2} ( \dot{u}_i^{(p)} - \dot{u}_i^{(0)} )\Delta t \\
\end{equation}

The timestep is limited by several factors. Stability analysis of SPH yields the CFL condition 

\begin{equation}
\label{ch02_sph02_eq023}
\Delta t_{CFL} = \mathcal{C}_{CFL} \min_{i} \frac{h_i}{c_i}
\end{equation}

where c is a constant chosen between $\mathcal{C}_{CFL} = 0.1 \dots 0.4$. All other integrated variables require, that the their maximal change per timestep is not more than their own magnitude. In order to avoid arbitrarily small timestep when a variable goes to zero, a minimal value for such quantities is introduced. This yields for example for the smoothing length the timestep

\begin{equation}
\label{ch02_sph02_eq024}
\Delta t_{h} = \mathcal{C} \min_{i} \frac{\dot{h}_i}{h_i + h_{min}}
\end{equation}

Other such variables are specific internal energy $u_i$ or the density $\rho_i$ in case the continuity equation is integrated. 

The predictor-corrector scheme has the disadvantage of a global time step. Other integrators like the leap-frog integrator allows hierarchical time stepping, due to their inherent symmetry of the sub-steps. In most cases the time step is limited by the CFL-criterion \ref{ch02_sph02_eq023}, which depends on the smoothing length and the speed of sound. For an ideal gas at constant entropy and when using particles of a fixed mass, the time step depends on density as 

\begin{equation}
\label{ch02_sph02_eq025}
\Delta t_{CFL} \sim \frac{h}{c} \sim \frac{1}{\sqrt[3]{\rho}} \sqrt{ \frac{\rho}{p} } \sim \frac{1}{\sqrt[3]{\rho}} \frac{1}{\sqrt{u}} \sim \frac{1}{\rho^{2/3}}
\end{equation}

When particle density varies over several orders of magnitudes, so does the time step. This is the case for cosmological simulations, where the collapse of dilute collapse gas to hot dense clumps results in immensely different time steps. Only hierarchical schemes where the actual time steps between derivations are chosen in the same order of magnitude as the required time step allow to integrate such a particle set with reasonable computational effort.
For collision events in the gravity regime, the time steps don't vary too much. Solids have similar speed of sound and smoothing lengths, also under compression. The formation of dilute vapor is not an issue, because particles in vapor state have much larger smoothing lengths than solid length, while retaining a comparable speed of sound, resulting in an even larger time step. A hierarchical time stepping scheme is therefore not necessary for the simulations of collisions.


\subsection{miscible SPH}
Standard SPH interpolates all variables weighed according to density. This works very well for single-phase fluids, but introduces difficulties with multi-phase fluids with large density contrasts. In reality, density can be discontinous along boundaries, for example for fluid droplets embedded in a gas. 

\begin{figure}[htbp]
\begin{center}
\includegraphics[scale=0.7]{01miscible_prof}
\caption{default}
\label{ch02_fig01}
\end{center}
\end{figure}


% A: mixed states
% . B: boundaries

Big density contrasts normally arise, when two fluid with different equations of state interact and have different densities for the same pressure and thermodynamical state. This difference can be orders of magnitude big like for an atmosphere on top of a liquid or a solids. For material in the same phase the density difference is usually not so big, but can still be easily a factor of a magnitude for example for water ice and solid iron ($\rho_0 \approx 0.9 g/cm^3$ vs. $7.8 g/cm^3$).

The actual problem of standard SPH and density contrasts arises for fluids with rather stiff equations of state, such as fluids and solids. Even a small deviation from the zero-pressure $\rho_0$ density results in a large pressure. 

%TODO: lagrange code chapter
Figure \ref{ch02_fig01} demonstrates that with an actual radial profile of a two-fluid 3D sphere in equilibrium, this means that all particles are at rest. The structure models an early proto-Earth of 0.9 $0.9 \ME$ with a $30 wt\%$ core of iron and a $70 wt\%$ mantle of silicate ($\silc$). A thin black line indicates the equilibrium solution of a 1D Lagrangian code (see chapter haba). Ideally the SPH particles should follow this solution. The left plots show the structure calculated with standard SPH. Particles have the actual density and pressure within a small error at most radii, except for the boundary between the core and the mantle, where there are differences in density leading to a huge pressure deviation. The iron particles have a density obtained by interpolating between 

This is actually the case except for the boundaries 


The upper two plots show density, the lower two pressure. The thin black line indicates 


If two fluids now share a common interface,

% TODO: show particles rho vs. r and p vs. r for standard SPH and miscible SPH for a m1H vs. m3H dump

% problem with mixed state
Either particle masses are chosen in the same order of magnitude for both fluids, such that the particles densities vary 

%




Big density contrast present a problem to SPH. 

motivation: big density contrasts
advantages and disadvantages of SPH

%\end{landscape}

\subsection{solid state SPH}
SPH can also be used to model conti

\citep{Nyffeler2004}
integrating density
equations in 
basic problem: stiff equations of state require a non-smooth density 
advantages and disadvantages of SPH

§
%\begin{landscape}

\begin{table}[htdp]
\begin{center}
\begin{tabular}{|l|l|l|l|}
\hline
variant & standard  & integrated density & miscible \\
\hline \hline
\multirow{2}{3cm}{first SPH sums} & 
\multirow{2}{3cm}{$\rho_i = \sum_{j} m_j \Wij $} & 
\multirow{2}{3cm}{$\rho_i$ is integrated }  & 
$ \delta_i = \sum_{j} \Wij$ \\
 &
 &
 & 
$\rho_i = m_i \delta_i $ \\
\hline
\multirow{4}{3cm}{second SPH sums} &
\multicolumn{2}{|l|}{$ \nabla \vv_i = \sum_{j} \frac{m_j}{\rho_j} \dWij $} & 
$ \nabla \vv_i = \frac{1}{m_i} \sum_{j} \dWij $ \\
& 
\multicolumn{2}{|l|}{$ \frac{du_i}{dt} = \frac{1}{2} \sum_{j} m_{j} \vvij \Big( \frac{p_i}{\rho_i^2}  + \frac{p_j}{\rho_j^2} \Big) \dWij $} & 
$ \frac{du_i}{dt} = \frac{1}{2 m_i} \sum_{j} \vvij \Big( \frac{p_i}{\delta_i^2}  + \frac{p_j}{\delta_j^2} \Big)  \dWij $ \\
& 
\multicolumn{2}{|l|}{$\frac{d \vv_i }{dt} = - \sum_{j} m_j \Big( \frac{p_i}{\rho_i^2} + \frac{p_j}{\rho_j^2} \Big) \nabla W_{ij}$} & 
$\frac{d \vv_i }{dt} = - \frac{1}{m_i} \sum_{j} \Big( \frac{p_i}{\delta_i^2} + \frac{p_j}{\delta_j^2} \Big) \nabla W_{ij}$\\
&   
& 
$\frac{d \rho_i}{dt} = \rho_i \nabla \vv_i$ &
\\
\hline
% $ \nabla \vv_i = \sum_{j} \frac{m_j}{\rho_j} \dWij $ 

\hline
\end{tabular}
\caption{comparison of SPH variants}
\end{center}
\label{default}
\end{table}


\subsection{ANEOS}
tables, interpolating, iterating, issues\\
show isobars\\
entropy vs. energy integration\\

\subsection{SPH summary}

% TODO landspace table

%%
%   G R A V I T Y 
%%
\citep{Abel:2010p3297}
\citep{Barnes:1986p2853}
\citep{Monaghan:2005p2677}
\citep{Ott:2003p3727}
\citep{Price:2004p2613}
\citep{Solenthaler:2008p3720}
\citep{Springel:2003p3298}
\citep{Monaghan:1992ARAA..30..543M}

% 
% TODO: standard SPH vs. integrated density, surface tension issues, miscible SPH \\

\section{Gravity}

\subsection{N-Body problem}

\subsection{Barnes and Hut Tree}


%%
%   I M P L E M E N T A T I O N
%%





\section{Implementation}
\subsection{Organizing particles in Trees}
building a tree\\
deleting a tree\\
neighbour search\\
tree walks\\
accelerating a tree: show skip, next and parent pointers by simple tree examples\\

\begin{figure}[htbp]
\begin{center}
\includegraphics[scale=0.6]{cell_wiring.pdf}
\caption{Wiring scheme of an octree cell node a depth $n$ with particle childs in subvolume 1 and 6 and a cell node as child 3. Parent pointers allow going upwards in the tree, child pointers downwards. Following the next pointers results in a pre-order tree traversal, where taking the \it{skip} pointer skips a cells subtree. Note that a \it{next} pointer }
\label{fig02walks}
\end{center}
\end{figure}


\begin{figure}[htbp]
\begin{center}
\includegraphics[scale=0.6]{orderwalks.pdf}
\caption{Post-order vs. pre-order recursors}
\label{fig02walks}
\end{center}
\end{figure}

\subsection{Tree algorithms}

\subsection{Tree parallelization: shared vs. distributed memory}
intro: shared vs. distributed memory\\

SPHLATCH v1\\
parallelizing a particle code: distributed sums vs. ghosts approach\\
parallelizing a tree\\
show ghost approach\\

disadvantage: distributed memory machines loose importance, not good for parameter space searches, mutlicore machines are the future


SPHLATCH v2\\
shared memory approach\\
load balancing approach, how this could be memory-\\
show cost (gravity, NS, total) for giant impact\\
caching issues\\

building the tree:

mention future of parallel computing: CPU vs. GPU computing, stream computing
trees on GPUs

\section{Tests?}
shocktube
simple gravity tree

\section{Algorithms}
clump detection\\
FOF vs. potential, not parallelized\\
initial conditions: show theoretical vs. unevolved vs. evolved densities, show example of impact with standard SPH and miscible SPH (chondritic, use moon case), setting up a \SSC \\

\subsubsection{Neighbour search}

\subsection{setting up initial conditions}
use of spheres for simplicity reason

\citep{Barnes:1986p2853}
\citep{Monaghan:2005p2677}
\citep{Price:2004p2613}

\bibliographystyle{plainnat}
\bibliography{bibliography}


