\graphicspath{{./02figs/}}

\chapter{Methods}
Collision events are characterised by a 
range of dynamical 

%\subsection*{Introduction}
%\includepdf[pages={2-},fitpaper=true]{/tree_musings}

Motivation to use SPH, spatial dynamics of collisions, SPH particles are where the mass is (not entirely true for collisions, but necessary), free boundaries



Lagrangian nature vs. grid techniques, complexity of AMR vs. simplicity of SPH

Neighbour search is the hardest thing about SPH



In this chapter we will discuss, how collisions in the strengthless regime as appearing in planetary systems can be modelled.
The first section gives a short review the governing physics of such collisions and will provide us with a closed set of equations, namely the Navier-Stokes equations and the 
This set is far too complicated to be solved in an analytical way. 
Smoothed particle Hydrodynamics (SPH) is a method which discretises fluids with particles of finite volume and presents a framework to solve the equations describing the fluid.
The second section derives different SPH formulations of the Navier-Stokes equations and demonstrates the advantages and disadvantages of each formulation.

gravity section: 

The initial conditions and also the outcome of collisional events can be charaterized by a 

Smoothed particle hydrodynamics (SPH) is a numerical method which can discretize 

(insert a nice picture of equations, code lines and numbers here)

\newpage

\section{Fluid- and Thermodynamics}
A continuum model 


%%
%   F L U I D S
%%
\subsection{Theory of Fluids}

\begin{equation}
\label{ch02_fld01_eq001}
\frac{\partial \rho}{\partial t} + \rho \nabla \mathbf{v} = 0
\end{equation}

\begin{equation}
\label{ch02_fld01_eq002}
\rho \frac{\partial \mathbf{v}}{\partial t} + \rho ( \mathbf{v} \nabla ) \mathbf{v} = - \nabla p + \eta \Delta \mathbf{v} + ( \zeta + \eta ) \nabla ( \nabla \mathbf{v} )
\end{equation}

show inviscous version

The Lagrangian of a inviscous fluid is given by
\begin{equation}
\label{ch02_fld01_eq003}
\Lagr = \int \Big( \frac{1}{2} \rho \vv ^2 - \rho u \Big) dV
\end{equation}



\cite{Thorne:2008}


\subsection{SPH formalism}
The basic idea of SPH is to interpolate all continuum variables in space with an interpolant.
If our variable is $A(\rv)$ defined at all points in space $\rv$, we start with the basic equality

\begin{equation}
\label{ch02_sph01_eq001}
A(\rv) = \int  \delta(\rv - \rvs) A(\rvs) \ud \rvs
\end{equation}

We now replace the $\delta$-function with a smoothing kernel $W$, which has typical width $h$, the smoothing length, and which fulfils the following two criteria

\begin{equation}
\label{ch02_sph01_eq002}
\limhzero W(\rv - \rvs, h) = \delta(\rv - \rvs) \hspace{1.0cm}
\int W(\rv - \rvs, h) \ud \rvs = 1
\end{equation}

Equation \ref{ch02_sph01_eq001} can now be re-written as 

\begin{equation}
\label{ch02_sph01_eq003}
A(\rv) = \int  W(\rv - \rvs, h) A(\rvs) \ud \rvs + O(h^2)
\end{equation}

The variable $A(\rv)$ is now smoothed out over the characteristic scale $h$. By replacing the $delta$-function with a kernel of finite length, an error in the order of $h^2$ is introduced. Now comes the particle part: The continuum is replaced by a finite set of points (\emph{particles}), on which the quantities are defined. The integral can now be replaced by a sum with a finite number of summands:

\begin{equation}
\label{ch02_sph01_eq004}
A(\rv) \approx \sum_{j} W(\rv - \rvj) A_i V_{j} \hspace{1.0cm} V_{j} = \frac{m_j}{\rho_j}
\end{equation}

$V_{j}$ is the particle volume and is usually given assigning each particle a mass and dividing it by the density. The particles used in the summation are called \emph{neighbours} and are indexed here with $j$. For most purposes, the variable only needs to be known at particle position $i$. For convenience, we write the kernel used between two particles as $\Wij = W( \rvi - \rvj, h)$\\

Spatial derivatives are easily given by taking the partial derivative of the integral interpolant variable and replacing it again by the sum over all neighbours:

\begin{equation}
\label{ch02_sph01_eq005}
\nabla A(\rv_i) = \frac{\partial}{\partial \rv}\int A(\rvs) \ud \rvs \frac{\partial}{\partial \rv}\ W(\rv_i - \rvs, h) + O(h^2)
\approx \sum_{j} A_j V_{j} \nabla \Wij 
\end{equation}

Rotations of variables can be similarly obtained:
\begin{equation}
\label{ch02_sph01_eq006}
\nabla \times A_i \approx \sum_{j} A_i V_{j} (\nabla \times \Wij)
\end{equation}

Like in in grid based schemes, there are different ways of calculating spatial derivatives in SPH.
Depending on the actual application of the derivative, whether it is to get the velocity divergence or a pressure gradient, there are better suited variants which show better error behaviour. They will be presented shortly.


\subsection{standard SPH}
The original and most widely used version of SPH uses the density and particle mass to calculate the particle volumes ($V_j = m_j / \rho_j)$ and the density to interpolate all variables. The density is given by the simple particle sum

\begin{equation}
\label{ch02_sph02_eq001}
\rho_i = \sum_{j} m_j \Wij
\end{equation}

The velocity divergence $\nabla \vv$ could in principle be written as
\begin{equation}
\label{ch02_sph02_eq002}
\nabla \vv_i = \sum_{j} \frac{m_j}{\rho_j} \vv_j \nabla \Wij
\end{equation}

but this form has the disadvantage of being asymmetric and will ultimately lead to a non-conservative form for the linear momentum. It is better to apply the \emph{second golden rule} of SPH \citep{Monaghan:1992p3721} to re-write the velocity divergence term with the density placed inside the operators:

\begin{equation}
\label{ch02_sph02_eq003}
\nabla \vv = \frac{1}{\rho} \Big( \nabla (\rho \vv) - \vv \nabla \rho \Big)
\end{equation}

so that the corresponding SPH sum becomes 

\begin{equation}
\label{ch02_sph02_eq003}
\nabla \vv_i = \frac{1}{\rho_i} \sum_{j} \frac{m_j}{\rho_j} \rho_j \vv_j \Wij - \frac{1}{\rho_i} \vv_i \sum_{j} \frac{m_j}{\rho_j} \rho_j = \frac{1}{\rho_i} \sum_{j} m_j \vvij \Wij 
\end{equation}

with $\vvij = \vv_j - \vv_i$, which is a symmetric form for the velocity divergence. If we take the partial time derivative of \ref{ch02_sph02_eq001}, where only the particle positions in the Kernel depend explicitly on time, we get: 

\begin{equation}
\label{ch02_sph02_eq004}
\frac{\partial \rho_i}{\partial t}  = \sum_{j} m_j \Big( \frac{\partial}{\partial t} \rvij \Big) \nabla \Wij = \sum_{j} m_j \vvij \nabla \Wij 
\end{equation}

So this equation together with the velocity divergence equation \ref{ch02_sph02_eq003} fulfills the continuity equation \ref{ch02_fld01_eq001} for each particle.

advantages and disadvantages of SPH


\subsection{integrating density}
advantages and disadvantages of SPH

\subsection{miscible SPH}
advantages and disadvantages of SPH


\section{Gravity}

\subsection{N-Body problem}

\subsection{Barnes and Hut Tree}




\section{Implementation}

\cite{Abel:2010p3297}
\citep{Barnes:1986p2853}
\citep{Monaghan:2005p2677}
\citep{Ott:2003p3727}
\citep{Price:2004p2613}
\citep{Solenthaler:2008p3720}
\citep{Springel:2003p3298}


\subsection{SPH implementation used}
standard SPH vs. integrated density, vs.\\
deriving SPH sums from Navier-Stokes\\
surface tension issues, miscible SPH, show m1H vs. m3H\\
refer to price\\

\subsection{Time integration}


\subsection{ANEOS}
tables, interpolating, iterating, issues\\
show isobars\\
entropy vs. energy integration\\

\section{Implementation}
\subsection{Organizing particles in Trees}
building a tree\\
deleting a tree\\
neighbour search\\
tree walks\\
accelerating a tree: show skip, next and parent pointers by simple tree examples\\

\begin{figure}[htbp]
\begin{center}
\includegraphics[scale=0.6]{cell_wiring.pdf}
\caption{Wiring scheme of an octree cell node a depth $n$ with particle childs in subvolume 1 and 6 and a cell node as child 3. Parent pointers allow going upwards in the tree, child pointers downwards. Following the next pointers results in a pre-order tree traversal, where taking the \it{skip} pointer skips a cells subtree. Note that a \it{next} pointer }
\label{fig02walks}
\end{center}
\end{figure}


\begin{figure}[htbp]
\begin{center}
\includegraphics[scale=0.6]{orderwalks.pdf}
\caption{Post-order vs. pre-order recursors}
\label{fig02walks}
\end{center}
\end{figure}


\subsection{Tree parallelization: shared vs. distributed memory}
intro: shared vs. distributed memory\\

SPHLATCH v1\\
parallelizing a particle code: distributed sums vs. ghosts approach\\
parallelizing a tree\\
show ghost approach\\

disadvantage: distributed memory machines loose importance, not good for parameter space searches, mutlicore machines are the future


SPHLATCH v2\\
shared memory approach\\
load balancing approach, how this could be memory-\\
show cost (gravity, NS, total) for giant impact\\
caching issues\\

building the tree:

mention future of parallel computing: CPU vs. GPU computing, stream computing
trees on GPUs

\section{Algorithms}
clump detection\\
FOF vs. potential, not parallelized\\
initial conditions: show theoretical vs. unevolved vs. evolved densities, show example of impact with standard SPH and miscible SPH (chondritic, use moon case), setting up a \SSC \\

\subsubsection{Neighbour search}

\subsection{setting up initial conditions}
use of spheres for simplicity reason

\citep{Barnes:1986p2853}
\citep{Monaghan:2005p2677}
\citep{Price:2004p2613}

\bibliographystyle{plainnat}
\bibliography{bibliography}


